\documentclass{ieeeaccess}
\usepackage{cite}
\usepackage{amsmath,amssymb,amsfonts}
\usepackage{algorithmic}
\usepackage{graphicx}
\usepackage{textcomp}
\usepackage{hyperref}
\usepackage[table]{xcolor}
\usepackage{algorithm}
\usepackage{algorithmic}


\usepackage{bm}
\makeatletter
\AtBeginDocument{\DeclareMathVersion{bold}
\SetSymbolFont{operators}{bold}{T1}{times}{b}{n}
\SetSymbolFont{NewLetters}{bold}{T1}{times}{b}{it}
\SetMathAlphabet{\mathrm}{bold}{T1}{times}{b}{n}
\SetMathAlphabet{\mathit}{bold}{T1}{times}{b}{it}
\SetMathAlphabet{\mathbf}{bold}{T1}{times}{b}{n}
\SetMathAlphabet{\mathtt}{bold}{OT1}{pcr}{b}{n}
\SetSymbolFont{symbols}{bold}{OMS}{cmsy}{b}{n}
\renewcommand\boldmath{\@nomath\boldmath\mathversion{bold}}}
\makeatother

\def\BibTeX{{\rm B\kern-.05em{\sc i\kern-.025em b}\kern-.08em
    T\kern-.1667em\lower.7ex\hbox{E}\kern-.125emX}}

%Your document starts from here ___________________________________________________

\graphicspath{ {./images/} }

\begin{document}
\history{Date of publication xxxx 00, 0000, date of current version xxxx 00, 0000.}
\doi{10.1109/ACCESS.2024.0429000}

\title{Refining Software Clustering: The Impact of Code Co-Changes on Architectural Reconstruction}
\author{\uppercase{Stana Adelina Diana}\authorrefmark{1} and
\uppercase{Sora Ioana}\authorrefmark{2}
}

\address[1]{Computer Science and Engineering Department
”Politehnica” University of Timisoara, Romania (e-mail: stana.adelina.diana@gmail.com)}
\address[2]{Computer Science and Engineering Department
”Politehnica” University of Timisoara, Romania (e-mail: ioana.sora@cs.upt.ro)}

\markboth
{Author \headeretal: Preparation of Papers for IEEE TRANSACTIONS and JOURNALS}
{Author \headeretal: Preparation of Papers for IEEE TRANSACTIONS and JOURNALS}

\corresp{Corresponding author: Stana Adelina Diana (e-mail: stana.adelina.diana@gmail.com).}


\begin{abstract}
Version control systems are important for tracking and managing changes in software code, but they also offer information about the relationships between software entities. When multiple software entities (e.g., classes, interfaces) are modified together, it may indicate an underlying connection. These code co-changes represent a type of software dependency that can be used together with structural or lexical dependencies to improve the understanding of a system's architecture.

In this paper, we explore using code co-changes as input for software clustering for architectural reconstruction. Since structural dependencies are the most commonly used dependencies in software clustering, we investigate whether integrating them with code co-changes provides better results than using either dependency type alone.

Our experiments are applied to four open-source Java projects from GitHub. For each project, we apply three distinct clustering algorithms (Louvain, Leiden, and DBSCAN) and evaluate their performance using two clustering evaluation metrics. These metrics allow us to compare the effectiveness of clustering based on code co-changes alone or clustering that combines co-changes with structural dependencies, offering a better understanding of how these co-changes influence software architecture reconstruction.
\end{abstract}

\begin{keywords}
Architectural reconstruction, code co-changes, logical dependencies, software clustering, software dependencies, versioning system.
\end{keywords}

\titlepgskip=-21pt

\maketitle

\section{Introduction}
\label{sec:introduction}

Software systems often face a lack of documentation. Even if there was original documentation at the beginning of development, over the years it may become outdated or lost. Additionally, the original developers may leave the company, taking with them knowledge about how the software was designed. This situation challenges the teams when it comes to maintenance or modernization. In this context, recovering the system's architecture is essential. Understanding the system's architecture helps developers better evaluate and understand the nature and impact of changes they need to make. One technique to aid reconstructing the system architecture is software clustering. Software clustering involves creating cohesive groups (modules) of software entities based on their dependencies and interactions.

Among the dependencies that can be used for software clustering are structural dependencies (relationships between entities based on code analysis), lexical dependencies (relationships based on naming conventions),  and code co-changes/logical dependencies (relationships between entities extracted from the version control system), among others. 

In this paper, we assess whether using structural dependencies combined with logical dependencies can provide better results than using each type of dependency alone. The structural dependencies are used as they are extracted from static code analysis, and the logical dependencies are filtered co-changes from the version control system \cite{b15}. The reason behind filtering the co-changes and not using them as they are in the versioning system is to enhance their reliability and make them easier to combine with structural dependencies, as their number can far exceed the structural dependencies number \cite{b1}.

The following research questions guide our investigation:

\begin{itemize}
    \item \textbf{RQ1:} Does using structural dependencies (SD) combined with logical dependencies (LD) improve software clustering results compared to traditional approaches using only structural dependencies (SD)?
    \item \textbf{RQ2:} Can using only logical dependencies (LD) produce good software clustering results?
    \item \textbf{RQ3:} How do different filtering settings for logical dependencies (LD) impact clustering results, and which filtering settings provide the best performance?
\end{itemize}

To answer these research questions, we apply three different clustering algorithms (Louvain, Leiden, and DBSCAN) to each of the four open-source projects. We then evaluate the results using two metrics, MQ (Modularization Quality)\cite{b10} and MoJoFM (Move and Join e\textbf{F}fectiveness \textbf{M}easure)\cite{mojofm}. These metrics allow us to compare the effectiveness of using structural dependencies alone, logical dependencies alone, and a combination of both, providing a better understanding of how different dependency types and filtering settings impact the clustering results.


In Section \ref{sec:related_work}, we review the related work and previous studies that used various dependencies for software clustering and their metrics for evaluation.
Section \ref{sec:methodology_implementation} details the workflow and implementation of our approach, including the extraction and filtering of dependencies, and the clustering algorithm used.
The plan and results of our experiments on four open-source projects are presented in Section \ref{sec:experiment}. Section \ref{sec:evaluation} evaluates our results by using the Modularization Quality (MQ) metric and the MoJoFM metric. We also manually analyze some of the clustering solutions.
Finally, Section \ref{sec:conclusion} contains our conclusions and findings.

\subsection{Abbreviations and Acronyms}

The following abbreviations and acronyms are used throughout this article:

\begin{itemize}
    \item \textbf{LD}: Logical Dependencies
    \item \textbf{SD}: Structural Dependencies
    \item \textbf{MQ}: Modularization Quality
    \item \textbf{MoJoFM}: Move and Join Effectiveness Measure
\end{itemize}

\section{Related Work}
\label{sec:related_work}

Several studies have explored the use of different types of dependencies in software clustering, applying different algorithms to improve clustering results and using various metrics to evaluate the results obtained.

Corazza et al. \cite{b13}, \cite{corazza2} used lexical dependencies derived from code comments, class names, attribute names, and parameter names, applying Hierarchical Agglomerative Clustering (HAC) to group related entities. For evaluating the results, the authors used a metric based on MoJo distance and NED (Non-Extremity Cluster Distribution), which measures that the formed clusters are not too large or too small.

Andritsos and Tzerpos ~\cite{tzerpos1} used structural dependencies and nonstructural attributes, such as file names and developer names, and proposed the LIMBO algorithm, a hierarchical clustering algorithm for clustering software systems. To evaluate the output of the algorithm, they used the MoJo distance metric.

Anquetil et al.~\cite{b14} also used lexical information, including file names, routine names, included files, and comments. They applied an n-gram-based clustering approach to detect semantic similarities between entities and evaluated the results using precision and recall metrics.

Maletic and Marcus \cite{maletic} propose an approach to software clustering that uses semantic dependencies extracted using Latent Semantic Indexing (LSI), a technique for identifying similarities between software components. They apply the minimal spanning tree (MST) algorithm for clustering, and evaluate the results using metrics based on both semantic and structural information.

Wu et al. \cite{wu} conducted a comparative study of six clustering algorithms, using structural dependencies on five software systems. Four of the algorithms are based on agglomerative clustering, one on program comprehension patterns, and the one  algorithm is a customized version of Bunch \cite{b10}. The performance of these algorithms was evaluated using the MoJo metric and NED (Non-Extreme Distribution).

 Mancoridis and Mitchell ~\cite{b10}, \cite{b101}, \cite{bunch} developed the Bunch tool for software clustering and used structural dependencies as input. The tool applies clustering algorithms to the structural dependency graph and outputs the system's organization. For evaluation, the authors introduced the Modularization Quality (MQ) metric, which is described in more detail in Section \ref{subsec:mq}, and is also used in our current experiments as an evaluation metric.

%
Şora et al.~\cite{b12} applied various algorithms, including minimum spanning tree-based methods, metric-based clustering, and hierarchical clustering, using structural dependencies to reconstruct software architecture.

Prajapati et al.~\cite{b18} considered structural, lexical, and change-history dependencies in their approach. They applied a many-objective optimization algorithm to balance various quality attributes in the clustering solution, demonstrating the benefits of a comprehensive dependency analysis.

Silva et al.~\cite{b16} investigated the use of co-change dependencies and the Chameleon algorithm, an agglomerative hierarchical clustering method, to identify clusters based on these dependencies. Their findings suggest that incorporating historical change information can improve the relevance of the clustering solution.


%%%%%%%%%%%%%%%%%%%%%%%%%%%%%%%%%%%%%%%%%%%%%%%%%%%%%

\section{Structural and Logical Dependencies in software clustering}
\label{sec:dependencies}

Software clustering relies on various types of dependencies to identify relationships between software entities. Structural dependencies, have been mostly used due to their reliability \cite{b12}. However, recent research has started incorporating other types of dependencies besides structural dependencies \cite{b13}, \cite{b14}, \cite{b18}.  In this section, we will present an overview of structural and logical dependencies, focusing on how they are extracted and used in software clustering.

\subsection{Structural Dependencies}

Structural dependencies are important for understanding the architecture of a software system because they reveal how different modules interact at the code level. In our research, we extract structural dependencies using a tool from our previous work \cite{b4}. This tool analyzes the source code to identify various types of relationships between software entities and exports them in CSV format.

Not all structural dependencies have the same impact on the architecture and behavior of a software system. To reflect their importance, we assign weights to different types of dependencies. This weighting ensures that more important relationships have a greater influence on the clustering process.

The dependency types and weights were previously defined in related works on clustering \cite{b19}, \cite{b20}.

Table \ref{tab:structural_weights} shows the weights assigned to different categories of structural dependencies, as proposed in previous works.

\begin{table}[htbp]
\centering
\begin{tabular}{|c|l|}
\hline
\textbf{Weight} & \textbf{Dependency types} \\ 
\hline
4 & Interface realization \\
3 & Inheritance, parameter, return type, field, cast, type binding \\ 
2 & Method call, field access, instantiation \\ 
1 & Local variable \\ 
 \hline
\end{tabular}
\caption{Weights assigned to different structural dependency types. \cite{b20}}
\label{tab:structural_weights}
\end{table}

The weights are assigned based on the following considerations:

\textit{Weight 4 – Interface Realization:} Assigned the highest weight because it signifies a strong architectural relationship. Implementing an interface means classes are expected to provide specific functionalities.

\textit{Weight 3 – Inheritance, Parameter, Return Type, Field, Cast, Type Binding:} These dependencies represent significant connections between entities. They include inheritance relationships and shared data or types, which affect the behavior and properties of entities.

\textit{Weight 2 – Method Call, Field Access, Instantiation:} These indicate interactions between classes but are less impactful than higher weights. They involve using methods or fields of other classes or creating instances.

\textit{Weight 1 – Local Variable:} Given the lowest weight, local variables are the most basic level of interaction.



\subsection{Logical Dependencies}
\label{subsec:ld}

We refer to logical dependencies as the filtered co-changes between software entities. A co-change occurs when two or more software entities are modified together during the same commit in the version control system. Co-changes indicate that these entities are likely related or dependent on each other, directly or indirectly.

There is a degree of uncertainty associated with co-changes. Compared to structural dependencies, where the presence of a dependency is certain, co-changes are less reliable. For example, if the system was migrated from one version control system to another, the first commit will include all the entities from the system at that point in time. Should we consider all these entities as related to one another in this case? This would introduce false dependencies and reduce the likelihood of achieving accurate results when combining them with more reliable types of dependencies.

Even if we address the issue of the first commit, it can still happen that a developer resolves multiple unrelated issues in the same commit (even though this is not recommended by development processes).

To solve this problem, in our previous works, we refined some filtering methods to ensure that the co-changes that remain after filtering are more reliable and suitable for use with other dependencies or individually \cite{b4}, \cite{b5}, \cite{b6}. Based on our previous results, the filters we decided to use further in our research are the commit size filter and the strength filter. Both filters are used together, and the end result is the set of logical dependencies that we use to generate software clusters.

\subsubsection{Commit Size Filter}

The commit size filter filters out all co-changes that originate from commits that exceed a certain number of files.

We are interested in extracting dependencies from code commits that involve feature development or bug fixes because that is when developers change related code files. If multiple unrelated features or bug fixes are solved in a single commit, it will appear as though all the entities in those files are related, even if they are not.

One scenario where this issue arises is the first commit of a software system when it is ported from one versioning system to another. This commit will contain many changed code files, but these changes do not originate from any functionality change, so they generate numerous irrelevant co-changes for the system.

A similar scenario occurs with merge commits. A merge commit is created automatically when you perform a merge operation to integrate changes from one branch into another. After integration, all commits from the branch are added to the target branch, and on top of that, there is the merge commit containing all changes from the commits merged into a single commit. Since this commit contains only a merge of multiple smaller, related issues/features solved, it is better to gather information from the smaller commits rather than from the overall merge commit.
What both scenarios above have in common is the large number of files involved in the commits. Based on our previous research and measurements regarding the number of files involved in a commit, we chose to set a threshold of 20 files \cite{b4}, \cite{b5}. Therefore, all co-changes that originate from commits with more than 20 changed code files are filtered out.


Table \ref{tab:commit_statistics} presents the commit statistics for the studied projects. The columns represent the percentage of commits that modified under 5 files, between 5 and 10 files, between 10 and 20 files, and above 20 files. We can observe that most of the commits have under 5 files changed, with Apache Tomcat having more than 90\% of the commits with less than 5 files changed. On the opposite side, only a few commits involve more than 20 files changed, Hibernate ORM having the highest percentage at 8.39\%. Overall, filtering based on commit size does not significantly reduce the number of commits considered.

\begin{table}[ht]
    \centering
    \caption{Commit statistics for studied projects}
    \label{tab:commit_statistics}
    \begin{tabular}{|l|c|c|c|c|c|}
        \hline
	 \textbf{Project Name} & \multicolumn{4}{c|}{\textbf{{Number of files changed} }}  \\ 
	\cline{2-5}
         & \textbf{Under 5} & \textbf{5-10} & \textbf{10-20} & \textbf{Above 20} \\ \hline
        Apache Ant & 83.83\% & 7.50\% & 4.17\% & 4.50\% \\ 
        Apache Tomcat & 90.95\% & 5.44\% & 2.04\% & 1.58\%  \\ 
        Hibernate ORM & 71.74\% & 12.37\% & 7.50\% & 8.39\%  \\ 
        Gson & 83.63\% & 9.85\% & 3.70\% & 2.81\%  \\ \hline
    \end{tabular}
\end{table}


\subsubsection{Strength Filter}

This filter focuses on the reliability of the co-changes. If a pair of co-changing entities appears only once in the entire history of the system, it might be less reliable than a pair that appears more frequently.

Zimmermann et al. introduced the support and confidence metrics to measure the significance of co-changes \cite{b7}.

The \textit{support metric} of a rule $(A \rightarrow B)$ where A is the antecedent and B is the consequent of the rule, is defined as the number of commits (transactions) in which both entities are changed together.

The \textit{confidence metric} of $(A \rightarrow B)$, as defined in Equation \eqref{eq:confidence} , focuses on the antecedent of the rule and is the number of commits together of both entities divided by the total number of commits of (A).


\begin{equation}
\text{Confidence}(A \rightarrow B) = \frac{\text{Nr. of commits containing } A \text{ and } B}{\text{Nr. of commits containing } A}
\label{eq:confidence}
\end{equation}


The confidence metric favors entities that change less and more frequently together, rather than entities that change more with a wider variation of other entities.

Assuming that (A) was changed in 10 commits and, of these 10 commits, 9 also included changes to (B), the confidence for the rule $(A \rightarrow B)$ is 0.9. On the other hand, if (C) was changed in 100 commits and, of these 100 commits, 50 also included changes to (D), the confidence for the rule $(C \rightarrow D)$ is 0.5. Therefore, in this scenario, we would have more confidence in the first pair $(A \rightarrow B)$ than in the second pair $(C \rightarrow D)$, even though the second pair has more than five times more updates together.

To favor entities that are involved in more commits together, we calculated a \textit{system factor}. This system factor is the mean value of the support metric values for all entity pairs.

The system factor is multiplied with the calculated confidence metric value. In addition, since we plan to use the metric values as weights, together with the weights of the structural dependencies, we multiply by 100 to scale the metric value to be supraunitary, and we clip the results between 0 and 100.


We refer to this addition to the original calculation formula as strength metric, and it is defined in Equation \eqref{eq:strength}.

\begin{equation}
\text{strength}(A \rightarrow B) = \text{confidence}(A \rightarrow B) \times 100 \times \text{system factor} 
\label{eq:strength}
\end{equation}


\subsubsection{Filter Application Process}

The overall filter application process is illustrated in Fig. \ref{fig:filtering}.  We begin by extracting all co-changes from the versioning system, and the first filter applied is the commit size filter. The commit size filter has a strict threshold of 20 files, meaning that any co-changes from commits involving more than 20 files are filtered out.

The co-changes that remain after applying the commit size filter are then processed using the strength filter. The strength filter uses multiple thresholds, specifically 10 different thresholds. We start with a threshold of 10 and increment it by 10 until we reach a maximum value of 100. The reason for not using a fixed threshold is to assess how different strength thresholds affect our cluster generation. 

\begin{figure}[t!]
  \centering
  \includegraphics[width=\columnwidth]{filtering.png}
  \caption{ \textbf{Filter application process}}
  \label{fig:filtering}
\end{figure}

\subsubsection{Dependency Extraction and Filtering Tool}

To extract and filter the co-changes, we used a previously developed tool \cite{b4}. This tool takes as input the GitHub repository address and the threshold values for commit and strength filters. The tool clones the repository, downloads all commit diffs starting from the first commit, examines all files changed in each commit to identify which entities have changed in those files, and creates undirected co-change dependencies between all changed entities within a commit.

The commit size filter is applied to these undirected co-change dependencies, since the metric value for $(A \rightarrow B)$ is the same as for $(B \rightarrow A)$. For the strength filter, each co-change dependency is converted into a directed co-change dependency, so for each $(A \rightarrow B)$ dependency, we have both $(A \rightarrow B)$ and $(B \rightarrow A)$. This conversion is necessary because, as mentioned in the previous section, the confidence filter evaluates the antecedent of the rule. Thus, the metric value for $(A \rightarrow B)$ differs from the metric value for $(B \rightarrow A)$. 

The remaining dependencies after applying the filters are then exported to a CSV file for further use.

It is important to note that the strength metric is only used for filtering and is \textit{not considered as a weight} of the dependencies. The \textit{weight assigned to each dependency is the number of commits in which both entities were updated together}.


\subsubsection{Combining Structural and Logical Dependencies}

\Figure[t!](topskip=0pt, botskip=0pt, midskip=0pt)[width=\textwidth]{codegraph.png}
{ \textbf{Dependency Graph: Combining structural and logical dependencies.}\label{fig:codegraph}}

When combining structural dependencies (SD) and logical dependencies (LD) in software clustering, both types of relationships are represented within the same graph.

Each entity in the system is represented as a node in the graph, and the dependencies between them are represented as directed, weighted edges.

\textit{SD and LD weights are combined} when the same pair of entities appears in both types of dependencies. In this case, the weights from SD and LD are summed, giving more influence to those entity pairs. When a pair of entities appears only in SD or only in LD, the edge is added to the graph together with its corresponding weight.

Figure \ref{fig:codegraph} illustrates the process of combining structural and logical dependencies in the same dependency graph. The structural dependencies between \texttt{House}, \texttt{OrangeCat}, and \texttt{CatBehavior} entities are visible from the source code analysis, showing relationships like interface implementation and method calls.

However, the combination of SD and LD reveals additional insights. One important observation is the logical dependency between \texttt{House} and \texttt{OrangeCat}, which is not observed from the structural analysis. This connection is extracted from version control and is further filtered using a 60\% strength filter. The filter reveals that \texttt{House} and \texttt{OrangeCat} have a significant co-change strength of 75.0, suggesting a strong logical relationship.

When SD and LD overlap, such as between \texttt{OrangeCat} and \texttt{CatBehavior}, their weights are summed. This summation process highlights relationships that are both structurally and logically significant.



%%%%%%%%%%%%%%%%%%%%%%%%%%%%%%%%%%%%%%%%%%%%%%%%%%%%%


\section{Methodology and Implementation}
\label{sec:methodology_implementation}

In this section, we present the methodology used to evaluate the impact of logical dependencies on the quality of software clustering solutions.

First, we describe the clustering algorithms used in our experiments: Louvain, Leiden, and DBSCAN. Next, we introduce the evaluation metrics used to assess the quality of the clustering results. Finally, we present the workflow and implementation of the tool developed for this research, which is built to process structural and logical dependencies, apply the selected clustering algorithms, and compute the evaluation metrics.

\subsection{Clustering Algorithms}
\subsubsection{Louvain}
\label{subsubsec:louvain}

The Louvain algorithm was originally developed by Blondel et al. and is used for finding community partitions (clusters) in large networks. The algorithm begins with a weighted network of N nodes, initially assigning each node to its own cluster, resulting in N clusters. For each node, the algorithm evaluates the modularity gain from moving the node to the cluster of each of its neighbors. Based on the results, the node is moved to the cluster with the maximum positive modularity gain. This process is repeated for all nodes until no further improvement in modularity is possible \cite{b8}, \cite{b9}.

\subsubsection{Leiden}
\label{subsubsec:leiden}

The Leiden algorithm, developed by Traag et al., is an improvement over the Louvain algorithm for community detection in large networks. Like Louvain, the Leiden algorithm begins with each node assigned to its own cluster and iteratively moves nodes between clusters to optimize modularity. However, the Leiden algorithm addresses some problems of the Louvain method, particularly regarding badly connected communities and runtime performance issues \cite{leiden}.

The Leiden algorithm introduces a refinement phase that ensures communities are locally optimally clustered and well-connected. This refinement step distinguishes the Leiden algorithm from Louvain.


\subsubsection{DBSCAN}
\label{subsubsec:dbscan}

The Density-Based Spatial Clustering of Applications with Noise (DBSCAN) algorithm, introduced by Ester et al., is a density-based clustering algorithm for identifying clusters of arbitrary shape and detecting noise in data \cite{dbscan}.

DBSCAN operates based on two main parameters:

\begin{itemize}
    \item \textbf{Eps}: It defines the radius within which to search for neighboring points.
    \item \textbf{MinPts}: The minimum number of points required for a dense region. It determines the minimum number of neighbors a point must have to be considered a core point.
\end{itemize}

The algorithm classifies points into three categories:

\begin{enumerate}
    \item \textbf{Core Points}: Points that have at least \textit{MinPts} neighbors within a radius of \textit{Eps}. These points are located in the interior of a cluster.
    \item \textbf{Border Points}: Points that have fewer than \textit{MinPts} neighbors within a radius of \textit{Eps} but are in the Eps-neighborhood of a core point. They are located on the edge of a cluster.
    \item \textbf{Noise}: Points that are neither core points nor border points.
\end{enumerate}

The DBSCAN algorithm starts by visiting an arbitrary point in the dataset. If the point is a core point, the algorithm starts a new cluster and retrieves all points that are reachable from this core point. All points are then marked as part of the cluster. If the point is a border point, it moves to the next point in the dataset. This process is repeated until all points have been visited.

For software clustering, DBSCAN can be applied by considering software entities as data points. A distance measure based on dependency weights can be used to compute the neighborhood between entities.

\subsection{Clustering Result Evaluation}
\label{subsec:evaluation_def}

To evaluate the clustering results, we use two metrics: the Modularity Quality (MQ) metric and the Move and Join Effectiveness Measure (MoJoFM) metric. Each of these metrics provides a different perspective on the quality of the clustering solutions. 

\subsubsection{Modularity Quality Metric}
\label{subsec:mq}

Mancoridis et al. introduced the Modularity Quality (MQ) metric to evaluate the modularization quality of a clustering solution based on the interaction between modules (clusters) \cite{b101},\cite{b10}. It evaluates the difference between connections within clusters and connections between different clusters.

The MQ of a graph partitioned into \( k \) clusters, where \( A_i \) is the Intra-Connectivity of the \( i \)-th cluster and \( E_{ij} \) is the Inter-Connectivity between the \( i \)-th and \( j \)-th clusters, is calculated using Equation \eqref{eq:mq} \cite{b2}.

\begin{equation}
MQ = \left( \frac{1}{k} \sum_{i=1}^{k} A_i \right) - \left( \frac{1}{k(k-1)} \sum_{i,j=1}^{k} E_{ij} \right)
\label{eq:mq}
\end{equation}

The MQ metric's value ranges between -1 and 1. A value of -1 means that the clusters have more connections between the clusters than within the clusters, while a value of 1 means that there are more connections within clusters than between clusters. A good clustering solution should have an MQ value close to 1, since this indicates that the clusters are more cohesive internally and have fewer connections to other clusters.

The MQ metric is useful because it does not require any additional input besides the clustering result itself. It relies on the structure of the clustered entities and their interactions. 

\subsubsection{MoJoFM Metric}
The MoJoFM metric was introduced by Wen and Tzerpos to evaluate the similarity between two different software clustering results \cite{mojofm}. The metric is based on the MoJo metric, which measures the absolute minimum number of \textit{Move} and \textit{Join} operations required to transform one clustering solution into another \cite{b3}, \cite{b11}. However, MoJoFM provides a similarity measure ranging between 0\% and 100\%, where 100\% indicates identical clustering solutions.

The MoJoFM metric is calculated using Equation \eqref{eq:mojofm}:

\begin{equation} 
\text{MoJoFM}(A, B) = \left(1 - \frac{\text{mno}(A, B)}{\max(\text{mno}(\forall A, B))}\right) \times 100\%
\label{eq:mojofm}
\end{equation}

where:

\begin{itemize}
    \item $\text{mno}(A, B)$ is the minimum number of \textit{Move} and \textit{Join} operations required to transform clustering solution $A$ into clustering solution $B$.
    \item $\max(\text{mno}(\forall A, B))$ is the maximum possible number of such operations required to transform any clustering $A$ into clustering $B$.
\end{itemize}

To use the metric, we first need to generate a reference clustering solution for comparison. This reference is manually created based on our analysis of the codebase.

By using the MoJoFM metric, we can evaluate the similarity between the generated clustering solutions and the reference clustering solution. We consider this metric useful when combining multiple types of dependencies because it clearly measures the similarity between the obtained clustering solutions and the same reference.


\subsection{Workflow for Software Clustering and Evaluation}
\label{subsec:tool_workflow}

\Figure[t!](topskip=0pt, botskip=0pt, midskip=0pt)[width=\textwidth]{workflow-tool.png}
{ \textbf{Tool workflow overview: input, processing and output.}\label{fig:tool}}

To achieve our goal of evaluating how the quality of clustering solutions is impacted by logical dependencies, we developed a Python tool capable of using any type of dependency, either alone or combined with other types of dependencies, as long as they are provided in CSV format. The tool clusters and evaluates software clustering solutions using either the MQ metric or the MoJoFM metric.  The entire workflow of the tool is presented in Fig. \ref{fig:tool}.



\subsubsection{Input}

The tool takes as input one or multiple dependency CSV files and the reference solution required for the MoJoFM metric. We designed the tool to accept multiple dependency files so that we can generate clustering solutions based on either a single type of dependency (structural or logical) or a combination of both.

Since the MoJoFM metric requires a reference solution to evaluate the obtained clustering solutions, we manually inspected the code and created reference clustering solutions, which we then provide as input for the tool.

\subsubsection{Processing}

The dependencies are saved in the CSV file in the following format: \texttt{antecedent of a dependency, consequent of a dependency, weight}. The tool reads each line, adds the antecedent and consequent as nodes in a directed graph, and creates an edge between them, with the weight from the CSV file becoming the edge weight. If multiple dependency files are processed and the same dependency is found in multiple files, the edge weights are summed.

The workflow of applying the clustering algorithms and performing the evaluations is shown in Algorithm \ref{alg:clustering_evaluation}. After all dependencies are read, the directed graph is passed to the clustering algorithms: Louvain, Leiden, and DBSCAN. Each algorithm generates its own clustering result. The results from each algorithm are then evaluated using the MQ metric and the MoJoFM metric. The MQ metric requires the directed graph and the clustering result, while the MoJoFM metric requires the reference clustering solution provided as input and the clustering result. 


\begin{algorithm}
\caption{Clustering and Evaluation Workflow}
\label{alg:clustering_evaluation}
\begin{algorithmic}[1]

\FOR{each clustering algorithm $A$ in $\{Louvain, Leiden, DBSCAN\}$}
    \STATE Apply algorithm $A$ to graph $G$ to generate clustering result $C$
    \STATE Compute the number of clusters from clustering result $C$
    \STATE Evaluate $C$ using the MQ metric with graph $G$ and result $C$
    \STATE Evaluate $C$ using the MoJoFM metric with reference solution $R$ and result $C$
\ENDFOR

\end{algorithmic}
\end{algorithm}

\subsubsection{Output}

After applying each clustering algorithm and completing both evaluations, we export the clustering result, the number of clusters from the clustering solution, and the values for both the MQ and MoJoFM metrics.


%%%%%%%%%%%%%%%%%%%%%%%%%%%%%%%%%%%%%%%%%%%%%%%%%%%%%%%%%%

\section{Experimental Plan and Results}
\label{sec:experiment}


\subsection{Experimental Plan}
\label{subsec:plan}

\subsubsection{Overview of projects used}

\begin{table*}
\centering
\caption{Overview of projects used in experimental analysis}
\label{tab:project_info}
\setlength{\tabcolsep}{7pt} 
\begin{tabular}{|l|l|c|l|p{4.7cm}|}
\hline
\textbf{Project Name} & \textbf{Release Tag} & \textbf{Commits} & \textbf{GitHub Repository Link} & \textbf{Repository Description} \\ 
 &  & \textbf{Number} & &  \\ \hline
Apache Ant & \href{https://github.com/apache/ant/tree/rel/1.10.13}{rel/1.10.13} & 14917 & \href{https://github.com/apache/ant}{https://github.com/apache/ant} & Apache Ant is a Java-based build tool. \\ \hline
Apache Tomcat & \href{https://github.com/apache/tomcat/tree/8.5.93}{8.5.93} & 22698 & \href{https://github.com/apache/tomcat}{https://github.com/apache/tomcat} & Apache Tomcat software powers numerous large-scale web applications across a diverse range of industries and organizations. \\ \hline
Hibernate ORM & \href{https://github.com/hibernate/hibernate-orm/tree/6.2.14}{6.2.14} & 16609 & \href{https://github.com/hibernate/hibernate-orm}{https://github.com/hibernate/hibernate-orm} & Hibernate ORM is an object/relational mapping solution for Java. \\ \hline
Gson & \href{https://github.com/google/gson/tree/gson-parent-2.10.1}{gson-parent-2.10.1} & 1772 & \href{https://github.com/google/gson}{https://github.com/google/gson} & A Java serialization/deserialization library to convert Java Objects into JSON and back. \\ \hline
\end{tabular}
\end{table*}

In Table \ref{tab:project_info}, we have synthesized all the information about the four projects used in our experiments. The 'Project Name' column contains the names of the software projects sourced from GitHub. The 'Release Tag' column contains the specific release tag of the project that was analyzed. For logical dependency extraction, we processed all the commits starting from the first commit up to the commit associated with the specified tag. For structural dependencies, we extracted the dependencies from the code of that specific tag. The 'Number of Commits' column provides the total number of commits used for logical dependencies extraction. The 'GitHub Repository Link' column includes the URL link to the project's repository on GitHub. Finally, the 'Repository Description' column provides a brief description of the project's purpose and functionality.

We mostly chose projects with more than 10,000 commits in their commit history, so that the logical dependencies extraction can be done on a larger information base. However, we selected 'Gson', which has a relatively small commit history (1,772 commits), to determine if our experiments work with a smaller information base.

\subsubsection{Tool runs}

To assess the impact of logical dependencies and to answer the research questions from section \ref{sec:introduction}, we run the tool presented in Section \ref{subsec:tool_workflow} in three different scenarios for all the projects from table \ref{tab:project_info}. All three scenarios are illustrated in Fig. \ref{fig:scenatrio}.

In the first scenario, we run the tool once, providing only the structural dependencies of the system as input for the clustering algorithm.

In the second scenario, we run the tool ten times, using only logical dependencies as input. We perform ten runs because we generate logical dependencies with different threshold values for the strength filter. We start with a threshold of 10 and increase it in steps of 10 up to 100, where 100 is the maximum value for the threshold.

In the third scenario, we combine logical with structural dependencies. Similar to the second scenario, we run the tool ten times, each time using structural dependencies and logical dependencies generated with different strength thresholds.

\begin{figure}[t!]
  \centering
  \includegraphics[width=\columnwidth]{scenario.png}
  \caption{ \textbf{Experimental scenarios for analyzing the impact of logical dependencies on clustering quality}}
  \label{fig:scenatrio}
\end{figure}

\subsection{Results}
\label{subsec:results}

\subsubsection{Detailed Results}

The experimental results are presented in this subsection in four tables, each corresponding to a different project. Table \ref{tab:clustering_results_ant} presents the results for Apache Ant, Table \ref{tab:clustering_results_tomcat} presents the results for Apache Tomcat, Table \ref{tab:clustering_results_hibernate} presents the results for Hibernate ORM, and Table \ref{tab:clustering_results_gson} presents the results for Gson.

Each table includes the following columns:
\begin{itemize}
\item \textbf{Dependency Type:} The types of dependencies used are as follows: SD for Structural Dependencies, LD for Logical Dependencies, and SD+LD for their combination. The strength threshold used is specified in parentheses right after LD.
\item \textbf{Entities Count:} The total number of software entities (such as classes, interfaces, enums) involved in clustering.
\item \textbf{System Cover:} Considering that the total number of entities extracted from the code base is the maximum number of entities in the system (first line of each table), we calculated the percentage of entities contained in the other inputs relative to the number of entities from the code base.
\item \textbf{Louvain/ Leiden/ DBSCAN:} The clustering algorithms used in the experiments.
\item \textbf{Nr. of Clusters:} The number of clusters from the clustering solution.
\item \textbf{MQ (Modularization Quality):} The result obtained when applying the MQ evaluation metric to the clustering solution.
\item \textbf{MoJoFM:} The result obtained when applying the MoJoFM evaluation metric to the clustering solution.
\end{itemize}

The rows in each table represent different dependency types and strength filter thresholds used in the clustering experiments.

\begin{table*}[htbp]
\centering
\caption{Clustering results based on different dependency types and strength filter thresholds for repository: \url{https://github.com/apache/ant}}
\label{tab:clustering_results_ant}
\setlength{\tabcolsep}{7pt} 
\begin{tabular}{|l|c|c|ccc|ccc|ccc|}
\hline
 \textbf{Dependency} &  \textbf{Entities} & \textbf{System} & \multicolumn{3}{c|}{\textbf{Louvain}} & \multicolumn{3}{c|}{\textbf{Leiden}} & \multicolumn{3}{c|}{\textbf{DBSCAN}} \\
\cline{4-12}
\textbf{Type (strength } &  \textbf{Count} & \textbf{Cover} & \textbf{Nr. of } & \textbf{MQ} & \textbf{MojoFM} & \textbf{Nr. of} & \textbf{MQ} & \textbf{MojoFM} & \textbf{Nr. of} & \textbf{MQ} & \textbf{MojoFM}  \\
\textbf{threshold)} &  & \textbf{(\%)} & \textbf{clusters} & & & \textbf{clusters} & &  & \textbf{clusters} & &\\
\hline
\rowcolor[HTML]{ECECEC} \textbf{SD} & 517 & 100.00 & 14 & 0.114 & 46.02 & 14 & 0.101 & 52.99 & 34 & 0.144 & 25.1  \\
\textbf{LD (10)} & 320 & 61.89 & 55 & 0.506 & 65.57 & 55 & 0.506 & 65.57 & 30 & 0.435 & 39.02 \\
\textbf{LD (20)} & 215 & 41.58 & 53 & 0.547 & 68 & 53 & 0.547 & 68 & 23 & 0.505 & 53.5 \\
\textbf{LD (30)} & 174 & 33.65 & 44 & 0.558 & 71.7 & 44 & 0.558 & 71.7 & 19 & 0.585 & 50 \\
\textbf{LD (40)} & 152 & 29.40 & 40 & 0.580 & 71.53 & 40 & 0.580 & 71.53 & 19 & 0.602 & 53.06 \\
\textbf{LD (50)} & 138 & 26.69 & 35 & 0.604 & 73.98 & 35 & 0.604 & 73.98 & 17 & 0.633 & 56.1 \\
\textbf{LD (60)} & 120 & 23.21 & 34 & 0.587 & 70.48 & 34 & 0.587 & 70.48 & 14 & 0.650 & 51.43 \\
\textbf{LD (70)} & 106 & 20.50 & 32 & 0.577 & 71.43 & 32 & 0.577 & 71.43 & 11 & 0.661 & 51.65 \\
\textbf{LD (80)} & 92 & 17.79 & 29 & 0.576 & 70.13 & 29 & 0.576 & 70.13 & 9 & \cellcolor[HTML]{FEF9E4}0.709 & 50.65 \\
\textbf{LD (90)} & 79 & 15.28 & 24 & 0.606 & 71.88 & 24 & 0.606 & 71.88 & 8 & 0.705 & 56.6 \\
\textbf{LD (100)} & 64 & 12.37 & 19 & \cellcolor[HTML]{fef9e4}0.611 & \cellcolor[HTML]{fef9e4}75.51 & 19 & \cellcolor[HTML]{fef9e4}0.611 & \cellcolor[HTML]{fef9e4}75.51 & 6 & 0.691 & \cellcolor[HTML]{fef9e4}56.93 \\
\hline
\textbf{SD+LD (10)} & 517 & 100.00 & 18 & \cellcolor[HTML]{fef9e4}0.355 & \cellcolor[HTML]{fef9e4}55.18 & 15 & 0.254 & \cellcolor[HTML]{fef9e4}54.98 & 37 & 0.147 & 25.9 \\
\textbf{SD+LD (20)} & 517 & 100.00 & 17 & 0.318 & 52.39 & 19 & \cellcolor[HTML]{fef9e4}0.365 & 53.78 & 32 & 0.149 & \cellcolor[HTML]{fef9e4}26.49 \\
\textbf{SD+LD (30)} & 517 & 100.00 & 16 & 0.282 & 53.19 & 16 & 0.265 & 54.78 & 30 & \cellcolor[HTML]{FEF9E4}0.159 & 24.5 \\
\textbf{SD+LD (40)} & 517 & 100.00 & 17 & 0.340 & 51.99 & 17 & 0.317 & 53.19 & 31 & 0.146 & 24.7 \\
\textbf{SD+LD (50)} & 517 & 100.00 & 15 & 0.248 & 52.59 & 19 & 0.298 & 56.77 & 31 & 0.146 & 24.7 \\
\textbf{SD+LD (60)} & 517 & 100.00 & 16 & 0.244 & 50.8 & 16 & 0.271 & 54.38 & 32 & 0.155 & 25.1 \\
\textbf{SD+LD (70)} & 517 & 100.00 & 15 & 0.238 & 51.00 & 18 & 0.281 & 52.99 & 32 & 0.155 & 25.1 \\
\textbf{SD+LD (80)} & 517 & 100.00 & 13 & 0.246 & 45.22 & 15 & 0.255 & 45.82 & 32 & 0.155 & 25.1 \\
\textbf{SD+LD (90)} & 517 & 100.00 & 14 & 0.258 & 46.02 & 16 & 0.268 & 47.01 & 32 & 0.155 & 25.1 \\
\textbf{SD+LD (100)} & 517 & 100.00 & 15 & 0.214 & 50.8 & 15 & 0.227 & 50.4 & 32 & 0.155 & 25.1 \\
\hline
\end{tabular}
\end{table*}



\begin{table*}[htbp]
\centering
\caption{Clustering results based on different dependency types and strength filter thresholds for repository: \href{https://github.com/apache/tomcat}{https://github.com/apache/tomcat}}
\label{tab:clustering_results_tomcat}
\setlength{\tabcolsep}{7pt} 
\begin{tabular}{|l|c|c|ccc|ccc|ccc|}
\hline
 \textbf{Dependency} &  \textbf{Entities} & \textbf{System} & \multicolumn{3}{c|}{\textbf{Louvain}} & \multicolumn{3}{c|}{\textbf{Leiden}} & \multicolumn{3}{c|}{\textbf{DBSCAN}} \\
\cline{4-12}
\textbf{Type (strength } &  \textbf{Count} & \textbf{Cover} & \textbf{Nr. of } & \textbf{MQ} & \textbf{MojoFM} & \textbf{Nr. of} & \textbf{MQ} & \textbf{MojoFM} & \textbf{Nr. of} & \textbf{MQ} & \textbf{MojoFM}  \\
\textbf{threshold)} &  & \textbf{(\%)} & \textbf{clusters} & & & \textbf{clusters} & &  & \textbf{clusters} & &\\
\hline
\rowcolor[HTML]{ECECEC} \textbf{SD} & 662 & 100.00 & 26 & 0.186 & 77.76 & 24 & 0.184 & 76.99 & 43 & 0.142 & 73.31  \\
\textbf{LD (10)} & 406 & 61.33 & 42 & 0.505 & 72.47 & 42 & 0.505 & 72.47 & 60 & 0.393 & 67.93 \\
\textbf{LD (20)} & 303 & 45.77 & 45 & 0.538 & 68.26 & 45 & 0.538 & 67.24 & 41 & 0.510 & 72.7 \\
\textbf{LD (30)} & 249 & 37.61 & 46 & 0.532 & 69.87 & 46 & 0.532 & 69.87 & 32 & 0.561 & 80.33 \\
\textbf{LD (40)} & 208 & 31.42 & 42 & 0.590 & 69.70 & 42 & 0.591 & 70.71 & 28 & 0.572 & 83.84 \\
\textbf{LD (50)} & 198 & 29.91 & 44 & 0.604 & 70.21 & 44 & 0.604 & 70.21 & 22 & 0.631 & 85.11 \\
\textbf{LD (60)} & 177 & 26.74 & 45 & 0.601 & 70.66 & 45 & 0.601 & 70.66 & 18 & 0.662 & 85.63 \\
\textbf{LD (70)} & 164 & 24.77 & 45 & 0.598 & 75.32 & 45 & 0.598 & 75.32 & 17 & 0.676 & 88.96 \\
\textbf{LD (80)} & 127 & 19.18 & 36 & 0.618 & 79.49 & 36 & 0.618 & 79.49 & 15 & 0.713 & \cellcolor[HTML]{fef9e4}89.74 \\
\textbf{LD (90)} & 116 & 17.52 & 32 & 0.623 & 81.13 & 32 & 0.623 & 81.13 & 14 & 0.718 & 89.62 \\
\textbf{LD (100)} & 110 & 16.62 & 30 & \cellcolor[HTML]{fef9e4}0.640 & \cellcolor[HTML]{fef9e4}85.00 & 30 & \cellcolor[HTML]{fef9e4}0.640 & \cellcolor[HTML]{fef9e4}85.00 & 13 & \cellcolor[HTML]{fef9e4}0.735 & 89.00 \\
\hline
\textbf{SD+LD(10)} & 662 & 100.00 & 28 & \cellcolor[HTML]{fef9e4}0.324 & 78.99 & 28 & 0.324 & 78.99 & 40 & 0.161 & \cellcolor[HTML]{fef9e4}74.23 \\
\textbf{SD+LD(20)} & 662 & 100.00 & 31 & 0.287 & 78.22 & 30 & 0.320 & \cellcolor[HTML]{fef9e4}80.06 & 50 & 0.189 & 73.31 \\
\textbf{SD+LD(30)} & 662 & 100.00 & 32 & 0.296 & \cellcolor[HTML]{fef9e4}79.92 & 32 & 0.277 & 75.77 & 45 & \cellcolor[HTML]{fef9e4}0.209 & 73.47 \\
\textbf{SD+LD(40)} & 662 & 100.00 & 34 & 0.292 & 79.91 & 32 & \cellcolor[HTML]{fef9e4}0.326 & 78.22 & 43 & 0.198 & 73.47 \\
\textbf{SD+LD(50)} & 662 & 100.00 & 33 & 0.294 & 76.53 & 35 & 0.301 & 76.23 & 43 & 0.196 & 73.31 \\
\textbf{SD+LD(60)} & 662 & 100.00 & 35 & 0.304 & 77.15 & 33 & 0.286 & 76.84 & 41 & 0.177 & 73.62 \\
\textbf{SD+LD(70)} & 662 & 100.00 & 34 & 0.282 & 76.69 & 34 & 0.292 & 77.45 & 41 & 0.166 & 73.62 \\
\textbf{SD+LD(80)} & 662 & 100.00 & 34 & 0.283 & 76.23 & 33 & 0.282 & 76.38 & 42 & 0.153 & 73.47 \\
\textbf{SD+LD(90)} & 662 & 100.00 & 31 & 0.311 & 78.99 & 31 & 0.311 & 78.99 & 43 & 0.153 & 73.31 \\
\textbf{SD+LD(100)} & 662 & 100.00 & 31 & 0.311 & 78.83 & 31 & 0.305 & 78.37 & 43 & 0.153 & 73.31 \\
\hline
\end{tabular}
\end{table*}




\begin{table*}[htbp]
\centering
\caption{Clustering results based on different dependency types and strength filter thresholds for repository: \href{https://github.com/hibernate/hibernate-orm}{https://github.com/hibernate/hibernate-orm}}
\label{tab:clustering_results_hibernate}
\setlength{\tabcolsep}{7pt} 
\begin{tabular}{|l|c|c|ccc|ccc|ccc|}
\hline
 \textbf{Dependency} &  \textbf{Entities} & \textbf{System} & \multicolumn{3}{c|}{\textbf{Louvain}} & \multicolumn{3}{c|}{\textbf{Leiden}} & \multicolumn{3}{c|}{\textbf{DBSCAN}} \\
\cline{4-12}
\textbf{Type (strength } &  \textbf{Count} & \textbf{Cover} & \textbf{Nr. of } & \textbf{MQ} & \textbf{MojoFM} & \textbf{Nr. of} & \textbf{MQ} & \textbf{MojoFM} & \textbf{Nr. of} & \textbf{MQ} & \textbf{MojoFM}  \\
\textbf{threshold)} &  & \textbf{(\%)} & \textbf{clusters} & & & \textbf{clusters} & &  & \textbf{clusters} & &\\
\hline
\rowcolor[HTML]{ECECEC} \textbf{SD} & 4414 & 100.00 & 30 & 0.09 & 52.23 & 23 & 0.071 & 52.44 & 373 & 0.128 & 46.32  \\
\textbf{LD (10)} & 1450 & 32.85 & 44 & 0.389 & 57.22 & 45 & 0.39 & 58.22 & 99 & 0.395 & 57.08 \\
\textbf{LD (20)} & 1325 & 30.02 & 66 & 0.397 & 62.66 & 66 & 0.397 & 62.66 & 151 & 0.378 & 63.36 \\
\textbf{LD (30)} & 1222 & 27.68 & 66 & 0.38 & 62.45 & 67 & 0.38 & 63.04 & 148 & 0.378 & 65.42 \\
\textbf{LD (40)} & 915 & 20.73 & 84 & 0.417 & 63.68 & 85 & 0.412 & 63.56 & 110 & 0.382 & 66.9 \\
\textbf{LD (50)} & 900 & 20.39 & 84 & 0.409 & 64.56 & 84 & 0.409 & 64.56 & 105 & 0.386 & 67.02 \\
\textbf{LD (60)} & 848 & 19.21 & 82 & 0.406 & 63.26 & 81 & 0.41 & 63.39 & 104 & 0.379 & 65.13 \\
\textbf{LD (70)} & 459 & 10.40 & 89 & 0.516 & \cellcolor[HTML]{fef9e4}69.08 & 89 & 0.516 & \cellcolor[HTML]{fef9e4}69.08 & 41 & 0.467 & 58.21 \\
\textbf{LD (80)} & 450 & 10.19 & 91 & 0.506 & 68.64 & 91 & 0.506 & 68.64 & 39 & 0.479 & \cellcolor[HTML]{fef9e4}60.49 \\
\textbf{LD (90)} & 432 & 9.79 & 92 & 0.492 & 66.93 & 92 & 0.492 & 66.93 & 40 & 0.473 & 58.66 \\
\textbf{LD (100)} & 356 & 8.07 & 81 & \cellcolor[HTML]{fef9e4}0.524 & 65.92 & 81 & \cellcolor[HTML]{fef9e4}0.524 & 65.92 & 29 & \cellcolor[HTML]{fef9e4}0.537 & 58.2 \\
\hline
\textbf{SD+LD (10)} & 4414 & 100.00 & 19 & 0.096 & 53.93 & 19 & 0.099 & 52.28 & 282 & 0.121 & 46.01 \\
\textbf{SD+LD (20)} & 4414 & 100.00 & 21 & 0.126 & 52.85 & 23 & 0.122 & 56.21 & 309 & \cellcolor[HTML]{fef9e4}0.135 & 47.4 \\
\textbf{SD+LD (30)} & 4414 & 100.00 & 26 & 0.121 & 55.76 & 26 & 0.15 & 54.54 & 317 & 0.135 & \cellcolor[HTML]{fef9e4}49.45 \\
\textbf{SD+LD (40)} & 4414 & 100.00 & 27 & \cellcolor[HTML]{fef9e4}0.182 & \cellcolor[HTML]{fef9e4}54.57 & 28 & \cellcolor[HTML]{fef9e4}0.163 & \cellcolor[HTML]{fef9e4}55.89 & 350 & 0.134 & 49.35 \\
\textbf{SD+LD (50)} & 4414 & 100.00 & 26 & 0.16 & 52.37 & 24 & 0.147 & 53.31 & 350 & 0.134 & 49.37 \\
\textbf{SD+LD (60)} & 4414 & 100.00 & 26 & 0.161 & 52.35 & 27 & 0.153 & 53.19 & 352 & 0.135 & 49.31 \\
\textbf{SD+LD (70)} & 4414 & 100.00 & 28 & 0.139 & 52.78 & 29 & 0.154 & 54.34 & 366 & 0.13 & 47.13 \\
\textbf{SD+LD (80)} & 4414 & 100.00 & 28 & 0.142 & 52.83 & 28 & 0.147 & 53.35 & 366 & 0.13 & 47.72 \\
\textbf{SD+LD (90)} & 4414 & 100.00 & 28 & 0.136 & 52.62 & 30 & 0.153 & 53.83 & 365 & 0.13 & 47.72 \\
\textbf{SD+LD (100)} & 4414 & 100.00 & 30 & 0.128 & 52.78 & 28 & 0.114 & 55.23 & 365 & 0.128 & 47.75 \\
\hline
\end{tabular}
\end{table*}






\begin{table*}[htbp]
\centering
\caption{Clustering results based on different dependency types and strength filter thresholds for repository: \href{https://github.com/google/gson}{https://github.com/google/gson}}
\label{tab:clustering_results_gson}
\setlength{\tabcolsep}{7pt} 
\begin{tabular}{|l|c|c|ccc|ccc|ccc|}
\hline
 \textbf{Dependency} &  \textbf{Entities} & \textbf{System} & \multicolumn{3}{c|}{\textbf{Louvain}} & \multicolumn{3}{c|}{\textbf{Leiden}} & \multicolumn{3}{c|}{\textbf{DBSCAN}} \\
\cline{4-12}
\textbf{Type (strength } &  \textbf{Count} & \textbf{Cover} & \textbf{Nr. of } & \textbf{MQ} & \textbf{MojoFM} & \textbf{Nr. of} & \textbf{MQ} & \textbf{MojoFM} & \textbf{Nr. of} & \textbf{MQ} & \textbf{MojoFM}  \\
\textbf{threshold)} &  & \textbf{(\%)} & \textbf{clusters} & & & \textbf{clusters} & &  & \textbf{clusters} & &\\
\hline
\rowcolor[HTML]{ECECEC} \textbf{gson SD} & 210 & 100.00 & 10 & 0.139 & 53.47 & 9 & 0.129 & 55.94 & 23 & 0.127 & 51.88 \\
\textbf{gson LD (10)} & 66 & 31.43 & 10 & 0.565 & 62.07 & 9 & 0.572 & 60.34 & 19 & 0.399 & 68.97 \\
\textbf{gson LD (20)} & 50 & 23.81 & 11 & 0.547 & 64.29 & 11 & 0.547 & 64.29 & 9 & 0.523 & 59.52 \\
\textbf{gson LD (30)} & 41 & 19.52 & 12 & 0.544 & 63.64 & 12 & 0.544 & 63.64 & 6 & 0.606 & 66.67 \\
\textbf{gson LD (40)} & 31 & 14.76 & 8 & \cellcolor[HTML]{fef9e4}0.635 & \cellcolor[HTML]{fef9e4}69.57 & 8 & \cellcolor[HTML]{fef9e4}0.635 & \cellcolor[HTML]{fef9e4}69.57 & 6 & \cellcolor[HTML]{fef9e4}0.612 & \cellcolor[HTML]{fef9e4}69.57 \\
\textbf{gson LD (50)} & 31 & 14.76 & 8 & 0.600 & 69.57 & 8 & 0.600 & 69.57 & 6 & 0.565 & 60.87 \\
\textbf{gson LD (60)} & 28 & 13.33 & 8 & 0.552 & 65.00 & 8 & 0.552 & 65.00 & 5 & 0.584 & 60.00 \\
\textbf{gson LD (70)} & 26 & 12.38 & 7 & 0.579 & 66.67 & 7 & 0.579 & 66.67 & 5 & 0.586 & 55.56 \\
\textbf{gson LD (80)} & 18 & 8.57 & 5 & 0.590 & 60.00 & 5 & 0.590 & 60.00 & 4 & 0.544 & 40.00 \\
\textbf{gson LD (90)} & 18 & 8.57 & 5 & 0.590 & 60.00 & 5 & 0.590 & 60.00 & 4 & 0.544 & 40.00 \\
\textbf{gson LD (100)} & 18 & 8.57 & 5 & 0.590 & 60.00 & 5 & 0.590 & 60.00 & 4 & 0.544 & 40.00 \\
\hline
\textbf{gson SD+LD(10)} & 210 & 100.00 & 11 & \cellcolor[HTML]{fef9e4}0.317 & \cellcolor[HTML]{fef9e4}64.36 & 11 & \cellcolor[HTML]{fef9e4}0.317 & \cellcolor[HTML]{fef9e4}64.36 & 20 & \cellcolor[HTML]{fef9e4}0.172 & \cellcolor[HTML]{fef9e4}63.86 \\
\textbf{gson SD+LD(20)} & 210 & 100.00 & 11 & 0.259 & 61.39 & 11 & 0.259 & 61.39 & 17 & 0.136 & 53.96 \\
\textbf{gson SD+LD(30)} & 210 & 100.00 & 11 & 0.277 & 61.39 & 11 & 0.277 & 61.39 & 20 & 0.136 & 55.94 \\
\textbf{gson SD+LD(40)} & 210 & 100.00 & 10 & 0.277 & 61.39 & 10 & 0.277 & 61.39 & 20 & 0.135 & 55.94 \\
\textbf{gson SD+LD(50)} & 210 & 100.00 & 10 & 0.270 & 60.40 & 11 & 0.270 & 60.89 & 20 & 0.135 & 55.94 \\
\textbf{gson SD+LD(60)} & 210 & 100.00 & 9 & 0.296 & 61.39 & 10 & 0.290 & 61.88 & 20 & 0.135 & 55.94 \\
\textbf{gson SD+LD(70)} & 210 & 100.00 & 8 & 0.295 & 59.41 & 8 & 0.295 & 59.41 & 20 & 0.135 & 55.94 \\
\textbf{gson SD+LD(80)} & 210 & 100.00 & 7 & 0.267 & 58.91 & 8 & 0.263 & 59.41 & 21 & 0.134 & 55.45 \\
\textbf{gson SD+LD(90)} & 210 & 100.00 & 7 & 0.267 & 58.91 & 7 & 0.267 & 58.91 & 21 & 0.134 & 55.45 \\
\textbf{gson SD+LD(100)} & 210 & 100.00 & 7 & 0.267 & 58.91 & 8 & 0.263 & 59.41 & 21 & 0.134 & 55.45 \\
\hline
\end{tabular}
\end{table*}





\subsubsection{Experiment on the impact of logical dependency weights}


\begin{table}[htbp]
\centering
\begin{tabular}{|c|c|c|c|c|}
\hline
\textbf{Dependency type} & \textbf{Ant} & \textbf{Tomcat} & \textbf{Hibernate} & \textbf{Gson} \\ \hline
\rowcolor[HTML]{ECECEC}\textbf{SD}     & 5.91  & 6.91  & 5.41  & 5.24  \\
\textbf{LD(10)} & 11.17 & 12.82 & 2.45  & 14.15 \\
\textbf{LD(20)} & 16.01 & 19.65 & 3.00  & 19.10 \\ 
\textbf{LD(30)} & 18.08 & 23.56 & 3.27  & 27.58 \\ 
\textbf{LD(40)} & 19.08 & 25.57 & 4.63  & 29.85 \\ 
\textbf{LD(50)} & 19.94 & 26.31 & 4.80  & 29.97 \\ 
\textbf{LD(60)} & 24.26 & 28.91 & 5.14  & 33.93 \\ 
\textbf{LD(70)} & 26.70 & 30.35 & 9.53  & 34.37 \\ 
\textbf{LD(80)} & 30.83 & 35.33 & 10.18 & 43.00 \\ 
\textbf{LD(90)} & 32.11 & 36.90 & 10.47 & 43.00 \\ 
\textbf{LD(100)} & 33.93 & 37.04 & 12.00 & 43.00 \\
 \hline
\end{tabular}
\caption{Average weights of Structural Dependencies (SD) and Logical Dependencies (LD).}
\label{tab:systems_weights}
\end{table}

We conducted an additional experiment to evaluate whether our weight calculation strategy for logical dependencies is suitable for scenarios where they are combined with the weights from structural dependencies.

Table \ref{tab:systems_weights} presents the average dependency weights across all the projects. The first row shows the average weights for structural dependencies, while the other rows show the average weights for logical dependencies at different strength thresholds.

It can be observed that the average weights for logical dependencies increase as the strength threshold rises. This led us to question whether we are assigning sufficient weight to the logical dependencies in our system.

Therefore, in this experiment, we increased the weight assigned to each logical dependency filtered with a 100\% strength threshold from the Apache Tomcat system and re-ran scenario III from Fig. \ref{fig:scenatrio}.

We wanted to see whether the weights assigned to logical dependencies are suitable to be combined with the weights from structural dependencies in the same directed graph by observing how the two metrics (MQ and MoJoFM) change once we increase the weights of the logical dependencies.


The results of this experiment can be found in Table \ref{tab:clustering_results_multiplication}. To maintain consistency, we used the same columns as in the other result tables (\ref{tab:clustering_results_ant}, \ref{tab:clustering_results_tomcat}, \ref{tab:clustering_results_hibernate}, \ref{tab:clustering_results_gson}), with the addition of two new columns:
\begin{itemize}
    \item \textbf{Multiplication Factor:} The value by which each logical dependency weight is multiplied.
    \item \textbf{Avg Weight:} The average weight assigned to each type of dependency used.
\end{itemize}

\begin{table*}[htbp]
\centering
\setlength{\tabcolsep}{7pt} 
\begin{tabular}{|l|c|c|ccc|ccc|ccc|}
\hline
\textbf{Multiplication} & \multicolumn{2}{c|}{\textbf{Avg. weight}} & \multicolumn{3}{c|}{\textbf{Louvain}} & \multicolumn{3}{c|}{\textbf{Leiden}} & \multicolumn{3}{c|}{\textbf{DBSCAN}} \\
\cline{4-12}
\textbf{Factor} & \textbf{SD} & \textbf{LD} & \textbf{Nr. of} & \textbf{MQ} & \textbf{MojoFM} & \textbf{Nr. of} & \textbf{MQ} & \textbf{MojoFM} & \textbf{Nr. of} & \textbf{MQ} & \textbf{MojoFM} \\
 &  &  & \textbf{clusters} & &  & \textbf{clusters} & && \textbf{clusters} &  & \\
\hline
1 & 6.91  & 37.04  & 31 & 0.311 & 78.83 & 31 & 0.305 & 78.37 & 43 & 0.153 & 73.31 \\ 
2 & 6.91  & 74.08  & 33 & 0.295 & 73.57 & 30 & 0.301 & 72.33 & 43 & 0.153 & 73.31 \\
3 & 6.91  & 111.12 & 34 & 0.313 & 74.19 & 33 & 0.309 & 72.80 & 43 & 0.153 & 73.31 \\ 
4 & 6.91  & 148.16 & 34 & 0.312 & 73.88 & 33 & 0.312 & 72.49 & 43 & 0.153 & 73.31 \\ 
5 & 6.91  & 185.20 & 34 & 0.306 & 73.88 & 33 & 0.308 & 72.18 & 43 & 0.153 & 73.31 \\ 
\hline
\end{tabular}
\caption{Impact of multiplication factors on clustering results for LD(100) in Apache Tomcat}
\label{tab:clustering_results_multiplication}
\end{table*}

\section{Evaluation}
\label{sec:evaluation}

The overall analysis of all the results from subsection \ref{subsec:results} indicates that combining structural and logical dependencies (SD+LD) provides better clustering solutions than using structural dependencies (SD) alone, covering 100\% of the system, meaning that no entity is missed during cluster generation. On the other hand, logical dependencies (LD) alone result in better clustering quality metrics compared to both SD and SD+LD, but they do not cover the entire system.

The best results for SD+LD are observed with a strength threshold between 10-40\%. For LD only, the best results are obtained at a 100\% strength threshold. The overall trend shows that for LD only, the MQ metric increases in value with a higher strength threshold, indicating more cohesive clusters, while the MoJo metric decreases, indicating that fewer transformations are needed to reach the expected clustering. For SD+LD, the best MQ and MoJo values are obtained at lower strength thresholds, and then both metrics indicate less effective clustering solution obtained with higher strength thresholds.

In the sections below, we analyze each project in detail.

\subsection{Analysis of clustering results}

\subsubsection{Apache Ant}

The clustering results for Apache Ant (Table \ref{tab:clustering_results_ant}) show that the combined structural and logical dependencies (SD+LD) achieved the best values with a strength threshold between 10\% and 30\%. The highest value for the MQ metric is reached with Leiden (MQ=0.365), at a strength threshold of 20\%, and the highest value for MoJoFM is also reached with Leiden (MoJoFM=54.98), at a strength threshold of 10\%.

Compared with the SD-only results, all SD+LD clustering solutions for all algorithms show better MQ metric values, with the highest MQ value for SD+LD being more than three times greater than the corresponding SD-only value. The MoJoFM metric does not always outperform the SD-only MoJoFM, but the values obtained are close to the SD-only results.

Logical dependencies (LD) alone produced the highest MQ and MoJoFM values at the 100\% strength threshold for both Leiden and Louvain, with the obtained metric values being higher than those of SD-only and SD+LD. However, the percentage of entities covered is significantly lower (LD(100) covers only 12.37\% of the system).
If we look at LD(10), where there is a 61.89\% coverage of the system, which is more compared to LD(100), both metrics still perform better than SD-only and SD+LD(10). However, there is still a gap until 100\% coverage.

From the clustering algorithm performance point of view, Leiden obtains the best evaluation metrics for all scenarios, followed by Louvain and DBSCAN.

An interesting observation is that, based on the LD-only results, where the metric results improve with a higher strength threshold, the SD+LD results do not follow the same pattern. On the contrary, the SD+LD metric results decline with a higher strength threshold. The explanation for this behavior may lie in the overlap between structural and logical dependencies. As presented in Figure \ref{fig:ant_correlation}, the number of LD decreases with a stricter strength threshold compared to the number of SD, and the overlap between the two types of dependencies increases.


\begin{figure}[t!]
  \centering
  \includegraphics[width=\columnwidth]{ant_correlation.png}
  \caption{\textbf{Apache Ant: Overlap between structural and logical dependencies and its correlation with clustering metrics.}}
  \label{fig:ant_correlation}
\end{figure}

In our previous works, we studied how these two types of dependencies overlap \cite{b4}, \cite{b5}. The reason behind those studies was to check how much new information we can get from using logical dependencies and how much is already present via structural dependencies.

Our overall findings were that with stricter filtering of logical dependencies, we obtain a higher percentage of overlap between the two dependencies, reaching at most 50\% of logical dependencies that are also structural dependencies.

So, we consider that the reason why SD+LD clustering solutions decline in performance with a higher strength threshold is that less and less new additional information is added to the system (logical dependencies that are not structural dependencies), causing the clustering solution to start resembling the performance of the SD-only solution. In Figure \ref{fig:ant_correlation}, we can see that LD(10) represents 61\% of the quantity of SD, while LD(100) is only at 12\%, with half of them being duplicated with SD.


\subsubsection{Apache Tomcat}

For Apache Tomcat (Table \ref{tab:clustering_results_tomcat}), the best results for SD+LD were obtained with strength thresholds between 10\% and 40\% across all algorithms. The Leiden algorithm achieved the best result for the MQ metric (MQ = 0.326) at a strength threshold of 40\%, while the best MoJoFM result was obtained at a threshold of 20\% (MoJoFM = 80.06), also with the Leiden algorithm. Compared with the SD-only results, the peak MQ values almost double the SD-only values. Similar to Apache Ant, the MQ values for all strength thresholds are higher than those for SD-only. While MoJoFM is not better for all thresholds, it still shows an improvement compared with the SD-only results.

The LD-only results show the highest MQ and MoJoFM values at LD(100), with an MQ of 0.640 and a MoJoFM of 85.00 for the Louvain and Leiden algorithms. However, as with the Apache Ant results, coverage remains an issue. LD(100) covers only 16.62\% of the system, lower than the coverage from SD-only or SD+LD combinations. On the other hand, LD(10), which covers 61.33\% of the system, still has better clustering solutions compared to SD-only, based on both MQ and MoJoFM results. 


We observe the same decline in results with a stricter strength threshold for SD+LD, and as with the previous system, these results can again be connected to the percentage of LD that also overlap with SD, as well as the decreasing number of LD compared to SD once the strength threshold becomes stricter. As shown in Figure \ref{fig:catalina_correlation}, LD filtered with a 10\% strength threshold overlaps with SD by approximately 22\%, while at a 100\% strength threshold, the overlap increases to approximately 39\%.



\begin{figure}[t!]
  \centering
  \includegraphics[width=\columnwidth]{catalina_correlation.png}
  \caption{\textbf{Apache Tomcat: Overlap between structural and logical dependencies and its correlation with clustering metrics.}}
  \label{fig:catalina_correlation}
\end{figure}


To ensure that the decline in performance for SD+LD with a stricter strength threshold is indeed caused by the fact that LD are significantly fewer compared to SD, and that part of them is duplicated by SD at higher thresholds, we added an additional experiment to our study. In this experiment, whose results can be found in Table \ref{tab:clustering_results_multiplication}, we increased the weights associated with LD(100) for Apache Tomcat to confirm that we are dealing with an LD quantity problem rather than a weight problem.

In Table \ref{tab:systems_weights}, which presents the average weights associated with the dependencies across all systems, we can see that for Tomcat, the average weight for LD(10) is already almost double than SD average weight. For LD(100), the average weight is approximately five times higher than that of SD. Still, we wanted to ensure that the issue comes from the LD quantity and not the weights, so we multiplied the LD weights by values ranging from 1 to 5, where 1 is the baseline (since multiplying by 1 does not change the LD weights).

Based on the metric values obtained for multiplication factors of 2 to 5, we can see that after increasing the weights assigned to LD, the metric values improve only slightly, with changes recorded at the second decimal: a 0.02 improvement for Louvain and 0.07 for Leiden. The results for DBSCAN remain unchanged due to the fixed values of MinPts and Eps and the already high LD weights for LD(100). 

From the experiment with weights, we can conclude that the issue is the quantity of dependencies. SD outnumbers LD, making LD information less impactful on the overall clustering solution.


\subsubsection{Hibernate ORM}



Hibernate ORM is the second largest system after Apache Tomcat in terms of the number of commits analyzed, with 16,609 commits considered for LD extraction. Additionally, it is the largest in terms of system size, with 4,414 entities (Table \ref{tab:project_info}).

Based on the results from Table \ref{tab:clustering_results_hibernate}, the SD+LD combination with a strength threshold of 40 performs best for this system. Louvain achieves the best MQ metric with a value of 0.182, while Leiden achieves the best MoJoFM metric with a value of 55.89. 

LD-only produced the best MQ values at 100\% strength and the best MoJoFM values at 70\% strength for both Louvain and Leiden. Compared to the previous systems, where both best values were recorded at the same strength threshold, Hibernate shows an earlier peak for MoJoFM. The system coverage is likely a factor contributing to this. Hibernate’s LD[100] covers only 8.07\% of the system, the lowest percentage among all systems studied. This low percentage can be linked to the number of commits compared to the number of entities. For Apache Tomcat, there were 22,698 commits and 662 entities, while for Hibernate, there were 16,609 commits and 4,414 entities. This indicates that not all entities had a chance to be updated in the version control system.

This observation is also reflected in Table \ref{tab:systems_weights}, where Hibernate has the lowest average weights for LD compared to SD across all systems. In other systems, LD(10) starts with almost double the average weight compared to SD, while Hibernate's LD(10) average weight is less than half of the SD average weight.

\begin{figure}[t!]
  \centering
  \includegraphics[width=\columnwidth]{hibernate_correlation.png}
  \caption{\textbf{Hibernate ORM: Overlap between structural and logical dependencies and its correlation with clustering metrics.}}
  \label{fig:hibernate_correlation}
\end{figure}

Hibernate also has the lowest overlap percentage between LD and SD, as shown in Figure \ref{fig:hibernate_correlation}. Similar to the other systems, the performance for MQ and MoJoFM decreases for SD+LD as the strength threshold becomes more stricter.

The results for Hibernate highlight the challenge of achieving better clustering in larger systems that have fewer commits relative to their size.



\subsubsection{Google Gson}


\begin{figure}[t!]
  \centering
  \includegraphics[width=\columnwidth]{gson_correlation.png}
  \caption{\textbf{Google Gson: Overlap between structural and logical dependencies and its correlation with clustering metrics.}}
  \label{fig:gson_correlation}
\end{figure}

Gson has the smallest number of commits analyzed, with 1,772 commits considered for LD extraction, and it is also the smallest in terms of system size, with 210 entities involved in clustering (Table \ref{tab:project_info}).

Based on the results from Table \ref{tab:clustering_results_gson}, the SD+LD combination with a strength threshold of 10 achieved the best MQ value of 0.317 for both Louvain and Leiden, and the best MoJoFM value of 64.36 for the same algorithms. Similar to Apache Ant and Tomcat, all SD+LD combinations achieve a better MQ value than SD-only.

LD-only produced the best MQ value at 40\% for both Louvain and Leiden, with an MQ of 0.635, and the best MoJoFM value at the same threshold for the same algorithms. It is the only system where the best MQ result for LD-only occurs at a lower strength threshold than 100\%. This is due to the very low number of entities remaining in the system at 100\% (only 18 out of 210).

In this particular system, it is more visible that in some scenarios, the Leiden clustering algorithm does not improve the Louvain algorithm. This observation is based on the fact that the values obtained for both MQ and MoJoFM metrics are the same in most cases for the Gson system for both algorithms.

It can also be observed that Gson has identical metric values for MQ and MoJoFM across multiple strength thresholds. Again, the small number of commits and the size of the system contribute to the stability of these metrics. 

Gson also has relatively high overlap rates between LD and SD compared to the other systems, as shown in Figure \ref{fig:gson_correlation}. Despite the constant values, the trend of decreasing performance for SD+LD with stricter strength filtering for LD is also present in Gson.

The results for this system highlight the difficulty of achieving better clustering solutions using logical dependencies in smaller systems with fewer commits. However, even for a small system like Gson, an improvement is still visible when using logical dependencies.

\subsection{Research Questions and Findings}

In this section, we will answer our research questions based on the results from our experiments. 

The following research questions guided our investigation:

\begin{itemize}
    \item \textbf{RQ1:} Does using structural dependencies (SD) combined with logical dependencies (LD) improve software clustering results compared to traditional approaches using only structural dependencies (SD)?
    \item \textbf{RQ2:} Can using only logical dependencies (LD) produce good software clustering results?
    \item \textbf{RQ3:} How do different filtering settings for logical dependencies (LD) impact clustering results, and which filtering settings provide the best performance?
\end{itemize}

\textbf{RQ1:} Based on the results from all four systems analyzed, the combination of SD and LD generally performed better than SD-only according to the MQ and MoJoFM metrics, confirming that using SD+LD improves software clustering results. Across different strength thresholds, SD+LD achieved higher MQ values for all clustering algorithms, indicating better modularization quality. In most cases, the MoJoFM metric also showed improvement, although not always across all LD strength thresholds. These results suggest that combining SD with LD leads to better clustering, as the logical dependencies indeed provide additional information about the relationships between software entities.


\textbf{RQ2:} Using LD-only also produced good clustering results, especially at higher strength thresholds. LD(100) produced the highest MQ and MoJoFM values for most systems. However, the coverage of LD-only is significantly lower at higher thresholds. Despite this, the metric results for LD(10) are, in most cases, still better than those for SD-only and SD+LD. For Ant and Tomcat, LD(10) covers more than 60\% of the system, while for Hibernate and Gson, the coverage is slightly above 30\%.


\textbf{RQ3:} The impact of different filtering settings on clustering performance was observed across all systems. For LD-only, lower strength thresholds, such as LD(10), provided better system coverage but resulted in lower MQ and MoJoFM values compared to higher thresholds like LD(100), where the best results were often achieved. The best performance for SD+LD was generally observed with strength thresholds between 10\% and 40\%, covering the entire system.


\section{Conclusion}
\label{sec:conclusion}
It is what it is

\begin{thebibliography}{00}

\bibitem{b1} Ajienka, Nemitari \& Capiluppi, Andrea. (2017). Understanding the Interplay between the Logical and Structural Coupling of Software Classes. Journal of Systems and Software. 134. 10.1016/j.jss.2017.08.042.
\bibitem{b2} S. Mancoridis, B. Mitchell, C. Rorres, Y. Chen, and E. Gansner, “Using automatic clustering to produce high-level system organizations of source code,” in Proceedings. 6th International Workshop on Program Comprehension. IWPC’98 (Cat. No.98TB100242), 1998, pp. 45–52.
\bibitem{b3} V. Tzerpos and R. C. Holt, "MoJo: a distance metric for software clusterings," Sixth Working Conference on Reverse Engineering (Cat. No.PR00303), Atlanta, GA, USA, 1999, pp. 187-193, doi: 10.1109/WCRE.1999.806959.
\bibitem{mojofm} Zhihua Wen and V. Tzerpos, "An effectiveness measure for software clustering algorithms," Proceedings. 12th IEEE International Workshop on Program Comprehension, 2004., Bari, Italy, 2004, pp. 194-203, doi: 10.1109/WPC.2004.1311061.
\bibitem{b4} Stana, Adelina-Diana \& Şora, Ioana. (2023). Logical dependencies: Extraction from the versioning system and usage in key classes detection. Computer Science and Information Systems. 20. 25-25. 10.2298/CSIS220518025S. 
\bibitem{b5} Stana, Adelina-Diana \& Şora, Ioana. (2019). Analyzing information from versioning systems to detect logical dependencies in software systems. 000015-000020. 10.1109/SACI46893.2019.9111582. 
\bibitem{b6} Stana, Adelina-Diana \& Şora, Ioana. (2019). Identifying Logical Dependencies from Co-Changing Classes. 486-493. 10.5220/0007758104860493. 
\bibitem{b7} T. Zimmermann, P. Weibgerber, S. Diehl and A. Zeller, "Mining version histories to guide software changes," Proceedings. 26th International Conference on Software Engineering, Edinburgh, UK, 2004, pp. 563-572, doi: 10.1109/ICSE.2004.1317478.
\bibitem{b8} Blondel, Vincent \& Guillaume, Jean-Loup \& Lambiotte, Renaud \& Lefebvre, Etienne. (2008). Fast Unfolding of Communities in Large Networks. Journal of Statistical Mechanics Theory and Experiment. 2008. 10.1088/1742-5468/2008/10/P10008. 
\bibitem{maletic} J. I. Maletic and A. Marcus, "Supporting program comprehension using semantic (lexical) and structural information," Proceedings of the 23rd International Conference on Software Engineering. ICSE 2001, Toronto, ON, Canada, 2001, pp. 103-112, doi: 10.1109/ICSE.2001.919085.
\bibitem{b9} Lancichinetti, Andrea \& Fortunato, Santo. (2009). Community Detection Algorithms: A Comparative Analysis. Physical review. E, Statistical, nonlinear, and soft matter physics. 80. 056117. 10.1103/PhysRevE.80.056117. 
\bibitem{leiden} Traag, V. \& Waltman, L. \& van Eck, Nees Jan. (2019). From Louvain to Leiden: guaranteeing well-connected communities. Scientific Reports. 9. 5233. 10.1038/s41598-019-41695-z. 
\bibitem{b10} S. Mancoridis, B. S. Mitchell, Y. Chen and E. R. Gansner, "Bunch: a clustering tool for the recovery and maintenance of software system structures," Proceedings IEEE International Conference on Software Maintenance - 1999 (ICSM'99). 'Software Maintenance for Business Change' (Cat. No.99CB36360), Oxford, UK, 1999, pp. 50-59, doi: 10.1109/ICSM.1999.792498.
\bibitem{wu} Wu, A. E. Hassan and R. C. Holt, "Comparison of clustering algorithms in the context of software evolution," 21st IEEE International Conference on Software Maintenance (ICSM'05), Budapest, Hungary, 2005, pp. 525-535, doi: 10.1109/ICSM.2005.31.
\bibitem{b101} S. Mancoridis, B. S. Mitchell, C. Rorres, Y. Chen and E. R. Gansner, "Using automatic clustering to produce high-level system organizations of source code," Proceedings. 6th International Workshop on Program Comprehension. IWPC'98 (Cat. No.98TB100242), Ischia, Italy, 1998, pp. 45-52, doi: 10.1109/WPC.1998.693283.
\bibitem{bunch} B. S. Mitchell and S. Mancoridis, "On the automatic modularization of software systems using the Bunch tool," in IEEE Transactions on Software Engineering, vol. 32, no. 3, pp. 193-208, March 2006, doi: 10.1109/TSE.2006.31. 
\bibitem{b11} Zhihua Wen and V. Tzerpos, "An effectiveness measure for software clustering algorithms," Proceedings. 12th IEEE International Workshop on Program Comprehension, 2004., Bari, Italy, 2004, pp. 194-203, doi: 10.1109/WPC.2004.1311061.
\bibitem{dbscan} Martin Ester, Hans-Peter Kriegel, Jörg Sander, and Xiaowei Xu. 1996. A density-based algorithm for discovering clusters in large spatial databases with noise. In Proceedings of the Second International Conference on Knowledge Discovery and Data Mining (KDD'96). AAAI Press, 226–231.
\bibitem{b12} Şora, Ioana. (2013). Software Architecture Reconstruction through Clustering: Finding the Right Similarity Factors. 45-54. 10.5220/0004599600450054. 
\bibitem{b13} A. Corazza, S. Di Martino, V. Maggio and G. Scanniello, "Investigating the use of lexical information for software system clustering," 2011 15th European Conference on Software Maintenance and Reengineering, Oldenburg, Germany, 2011, pp. 35-44, doi: 10.1109/CSMR.2011.8.
\bibitem{corazza2} Corazza, Anna \& Di Martino, Sergio \& Scanniello, Giuseppe. (2010). A Probabilistic Based Approach towards Software System Clustering. CSMR. 88 - 96. 10.1109/CSMR.2010.36. 
\bibitem{b14} Anquetil, Nicolas \& Lethbridge, Timothy. (1998). File Clustering Using Naming Conventions for Legacy Systems. 10.1145/782010.782012. 
\bibitem{b15} Oliva, Gustavo \& Gerosa, Marco Aurelio. (2012). A Method for the Identification of Logical Dependencies. Proceedings - 2012 IEEE 7th International Conference on Global Software Engineering Workshops, ICGSEW 2012. 70-72. 10.1109/ICGSEW.2012.19. 
\bibitem{b16} Silva, Luciana \& Valente, Marco \& Maia, Marcelo. (2015). Co-change Clusters: Extraction and Application on Assessing Software Modularity. Transactions on Aspect-Oriented Software Development. 10.1007/978-3-662-46734-3\_3. 
\bibitem{b17} Murtagh, Fionn \& Contreras, Pedro. (2012). Algorithms for hierarchical clustering: An overview. Wiley Interdisc. Rew.: Data Mining and Knowledge Discovery. 2. 86-97. 10.1002/widm.53. 
\bibitem{b18} Amarjeet Prajapati, Anshu Parashar, Amit Rathee. Multi-dimensional information-driven many-objective software remodularization approach. Frontiers of Computer Science in China, 17(3):173209, 2023.
\bibitem{b19} I. Şora, G. Glodean and M. Gligor, "Software architecture reconstruction: An approach based on combining graph clustering and partitioning," 2010 International Joint Conference on Computational Cybernetics and Technical Informatics, Timisoara, Romania, 2010, pp. 259-264, doi: 10.1109/ICCCYB.2010.5491289.
\bibitem{b20} Ioana Şora, Ciprian-Bogdan Chirila, “Finding Key Classes in Object-Oriented Software Systems by Techniques Based on Static Analysis”  Information and Software Technology, 2019.
\bibitem{tzerpos1} P. Andritsos and V. Tzerpos, "Information-theoretic software clustering," in IEEE Transactions on Software Engineering, vol. 31, no. 2, pp. 150-165, Feb. 2005, doi: 10.1109/TSE.2005.25. 
\end{thebibliography}

\begin{IEEEbiographynophoto}{First} Add text here
\end{IEEEbiographynophoto}

\begin{IEEEbiographynophoto}{Second} Add text here
\end{IEEEbiographynophoto}

\EOD

\end{document}
