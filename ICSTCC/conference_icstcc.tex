\documentclass[conference]{IEEEtran}
\IEEEoverridecommandlockouts
% The preceding line is only needed to identify funding in the first footnote. If that is unneeded, please comment it out.
\usepackage{cite}
\usepackage{amsmath,amssymb,amsfonts}
\usepackage{algorithmic}
\usepackage{graphicx}
\usepackage{textcomp}
\usepackage{xcolor}
\usepackage{booktabs}
\usepackage{soul}
\usepackage{url}
\def\BibTeX{{\rm B\kern-.05em{\sc i\kern-.025em b}\kern-.08em
    T\kern-.1667em\lower.7ex\hbox{E}\kern-.125emX}}
\begin{document}

\title{Logical dependencies: applications in software clustering} %Integrating Logical Dependencies in Software Clustering: A Path to Enhanced Architecture Reconstruction

\author{\IEEEauthorblockN{1\textsuperscript{st} Adelina Stana}
\IEEEauthorblockA{\textit{dept. name of organization (of Aff.)} \\
\textit{name of organization (of Aff.)}\\
City, Country \\
email address }
\and
\IEEEauthorblockN{2\textsuperscript{nd} Ioana Sora}
\IEEEauthorblockA{\textit{dept. name of organization (of Aff.)} \\
\textit{name of organization (of Aff.)}\\
City, Country \\
email address }
}

\maketitle

\begin{abstract}
This document is a model and instructions for \LaTeX.
This and the IEEEtran.cls file define the components of your paper [title, text, heads, etc.]. *CRITICAL: Do Not Use Symbols, Special Characters, Footnotes, 
or Math in Paper Title or Abstract.
\end{abstract}

\begin{IEEEkeywords}
component, formatting, style, styling, insert
\end{IEEEkeywords}

\section{Introduction}

The software architecture helps developers in gaining a better understanding of the system and its expected behavior. Additionally, it is also of great help in change management. By knowing the existing system architecture, project managers can assess whether a requested change can be easily implemented or not.

Architecture reconstruction appears in contexts where a software system lacks documentation entirely, or where the documentation hasn't been updated to reflect changes in the system. 


We intend to use connections obtained from the versioning system to enhance the results of clustering methods that previously relied solely on connections extracted from code.

\section{Logical dependencies}

About how LD are obtained; cite past papers

\section{Related work}

Talk about the results of others and then about what we intend to do.


\section{Method}
Explain how the script works; insert diagram 

\subsection{Clustering algorithms: MST Clustering}
Minimum Spanning Tree Clustering is a hierarchical clustering method based on the construction of a minimum spanning tree from the weighted graph formed by the data connections of the system \cite{mst_clustering}.


\subsection{Clustering algorithms: Louvain Clustering}
Louvain Clustering is a community detection algorithm designed for finding clusters or communities in complex networks. The Louvain method involves a greedy algorithm that moves nodes between clusters to obtain clusters that are highly interconnected \cite{louvain_clustering}.


\section{Results}

\begin{table}[htbp]
  \centering
  \caption{Comparison of Louvain and MST clustering results for LD}
  \label{tab:clustering-results1}
  \begin{tabular}{lc|cc|cc}
    \toprule
    \textbf{Dataset} & \textbf{Entities} & \multicolumn{2}{c}{\textbf{Cluster count}} & \multicolumn{2}{c}{\textbf{MQ metric}} \\
    & \textbf{Count} & \textbf{Louvain} & \textbf{MST} & \textbf{Louvain} & \textbf{MST} \\
    \midrule
    SD only & 517 & 14 & 228 & 0.085 & 0.084 \\
    LD strength 10\% & 517 & 272 & 353 & 0.047 & 0.031 \\
    LD strength 20\% & 517 & 355 & 360 & 0.04 & 0.037 \\
    LD strength 30\% & 517 & 387 & 404 & 0.036 & 0.033 \\
    LD strength 40\% & 517 & 405 & 413 & 0.034 & 0.033 \\
    LD strength 50\% & 517 & 414 & 422 & 0.03 & 0.029 \\
    LD strength 60\% & 517 & 431 & 439 & 0.029 & 0.027 \\
    LD strength 70\% & 517 & 443 & 444 & 0.027 & 0.026 \\
    LD strength 80\% & 517 & 454 & 460 & 0.024 & 0.022 \\
    LD strength 90\% & 517 & 462 & 463 & 0.021 & 0.021 \\
    LD strength 100\% & 517 & 472 & 480 & 0.019 & 0.016 \\
    \bottomrule
  \end{tabular}
\end{table}



\begin{table}[htbp]
  \centering
  \caption{Comparison of Louvain and MST clustering results for LD combined with SD}
  \label{tab:clustering-results2}
  \begin{tabular}{lc|cc|cc}
    \toprule
    \textbf{Dataset} & \textbf{Entities} & \multicolumn{2}{c}{\textbf{Cluster count}} & \multicolumn{2}{c}{\textbf{MQ metric}} \\
    & \textbf{Count} & \textbf{Louvain} & \textbf{MST} & \textbf{Louvain} & \textbf{MST} \\
    \midrule
    SD only & 517 & 14 & 228 & 0.085 & 0.084 \\
    SD and LD strength 10\% & 517 & 15 & 74 & 0.087 & \underline{0.054} \\
    SD and LD strength 20\% & 517 & 13 & 8 & 0.071 & \underline{0.059} \\
    SD and LD strength 30\% & 517 & 13 & 14 & 0.071 & \underline{0.083} \\
    SD and LD strength 40\% & 517 & 13 & 8 & 0.071 & 0.109 \\
    SD and LD strength 50\% & 517 & 13 & 8 & 0.071 & 0.109 \\
    SD and LD strength 60\% & 517 & 13 & 8 & 0.071 & 0.109 \\
    SD and LD strength 70\% & 517 & 13 & 10 & 0.071 & 0.155 \\
    SD and LD strength 80\% & 517 & 13 & 6 & 0.071 & 0.177 \\
    SD and LD strength 90\% & 517 & 13 & 2 & 0.071 & 0.129 \\
    SD and LD strength 100\% & 517 & 13 & 8 & 0.072 & 0.166 \\
    \bottomrule
  \end{tabular}
\end{table}




\section{Discussion}

Based on the results from table \ref{tab:clustering-results2}, we can observe that the combined approach of structural dependencies and logical dependencies gives a Modularity Quality (MQ) metric of 0.071, which is an improvement over the 0.085 MQ metric obtained when considering only structural dependencies. 

Next, we will examine the clustering solutions obtained only from structural dependencies, in comparison to the clustering solution derived from incorporating both structural and logical dependencies, filtered with a threshold of 20\% for strength.

The entities listed below are placed in different clusters when analyzed based on structural dependencies alone, compared with the analysis using both logical and structural dependencies: taskdefs.Available\$FileDir, taskdefs.Concat, taskdefs.Concat\$1, taskdefs.Concat\$MultiReader, taskdefs.Concat\$TextElement, taskdefs.Javadoc\$AccessType, util.WeakishReference, util.WeakishReference\$HardReference.

The placement of the Concat class and its inner classes (Concat\$1, Concat\$MultiReader, Concat\$TextElement) in a different cluster might be based on it's usage and purpose; according to the documentation : "This class contains the 'concat' task, used to concatenate a series of files into a single stream." \cite{ant_concat}.

\section{Conclusion & Future work}


\bibliographystyle{IEEEtran}
\bibliography{logicaldepd}

\end{document}
