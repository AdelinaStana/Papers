\documentclass{beamer}
\mode<presentation>
{
  \usetheme{default}      % or try Darmstadt, Madrid, Warsaw, ...
  \usecolortheme{default} % or try albatross, beaver, crane, ...
  \usefonttheme{default}  % or try serif, structurebold, ...
  \setbeamertemplate{navigation symbols}{}
  \setbeamertemplate{caption}[numbered]
} 

\usepackage[english]{babel}
\usepackage[utf8x]{inputenc}
\usepackage{scrextend}
\usepackage{graphicx}
\usepackage{booktabs}
\usepackage{adjustbox}
\usepackage{marvosym}
\graphicspath{ {images/} }

\title[Pres]{Methods and Tools for the Analysis of Legacy Software Systems}
\author{Stana Adelina Diana}
\institute{Computer Science and Engineering Department\\
"Politehnica" University of Timisoara}
\date{EIACSDR, 2019}

\begin{document}

\begin{frame}
  \titlepage
\end{frame}

%%%%%%%%%%%%%%%%%%%%%%%%%%%%%%%%%%%%%%%%%%

 \begin{frame}
\frametitle{Presentation of the research topic}
 The thesis will develop methods for the analysis of software systems
 using historical information from the versioning systems\footnote{Versioning systems keep track of every change to a file over time so early versions can be restored and used by software teams.}. 
\end{frame}

%%%%%%%%%%%%%%%%%%%%%%%%%%%%%%%%%%%%%%%%%%

 \begin{frame}
\frametitle{Structural dependencies}
\begin{block}{Definition}
Structural dependencies are the result of \it{source code analysis} and can be extracted from : members, call parameters, local variables. 
\end{block}

\begin{center}
     \begin{figure}
	\includegraphics[width=\textwidth]{structural_dep.png}
	\caption{\label{fig:fig}Example of structural dependency between two classes}
     \end{figure}
\end{center}

\end{frame}

%%%%%%%%%%%%%%%%%%%%%%%%%%%%%%%%%%%%%%%%%%%

 \begin{frame}
\frametitle{Logical dependencies}
\begin{block}{Definition}
 Logical dependencies are the result of software history analysis and can reveal relationships that are not present in the source code code (structural dependencies).
\end{block}

\begin{center}
     \begin{figure}
	\includegraphics[width=\textwidth]{fig1.png}
	\caption{\label{fig:fig1}Example of logical and structural dependencies}
     \end{figure}
\end{center}

\end{frame}

%%%%%%%%%%%%%%%%%%%%%%%%%%%%%%%%%%%%%%%%%%%

 \begin{frame}
\frametitle{Current status of research}
The current trend recommends that general dependency management methods and tools should also include logical dependencies besides the structural dependencies \footnote{Gustavo Ansaldi Oliva and Marco Aurelio Gerosa. On the interplay between
structural and logical dependencies in open-source software.}, \footnote{Nemitari Ajienka and Andrea Capiluppi. Understanding the interplay between the logical and structural coupling of software classes.}. \\
But there are no strict rules to \textit{filter co-changes into logical dependencies}, other researches filtered co-changes only in order to decrease their number and not to increase their validity.

\end{frame}

%%%%%%%%%%%%%%%%%%%%%%%%%%%%%%%%%%%%%%%%%%

 \begin{frame}
\frametitle{Research content - filter co-changing classes into logical dependencies}
\begin{center}
     \begin{figure}
	\includegraphics[width=\textwidth]{filter.jpg}
	\caption{\label{fig:fig3}Filters for co-changing classes.}
     \end{figure}
\end{center}
\footnote{Adelina Diana Stana and Ioana Sora. Identifying logical dependencies from co-changing classes. }

\end{frame}
%%%%%%%%%%%%%%%%%%%%%%%%%%%%%%%%%%%%%%%%%%

 \begin{frame}
\frametitle{Research content - refine filter for occurrences of co-changing classes}

\begin{center}
     \begin{figure}
	\includegraphics[width= 9.0 cm]{filter_occ.PNG}
	\caption{\label{fig:fig4} Occurrences rates of co-changing classes extracted from one system. }
     \end{figure}
\end{center}

\end{frame}
%%%%%%%%%%%%%%%%%%%%%%%%%%%%%%%%%%%%%%%%%%

 \begin{frame}
\frametitle{Research content - architectural reconstruction}
Use the logical dependencies extracted among structural dependencies in tools for architectural reconstruction to evaluate the improvement.

\end{frame}
%%%%%%%%%%%%%%%%%%%%%%%%%%%%%%%%%%%%%%%%%%

 \begin{frame}
\frametitle{Research content - software metrics}
Compare the number of logical dependencies with metrics and study their connections. Metrics:
\begin{itemize}
	\item  Fan Out - number of other classes referenced by a class.
	\item  Fan In - number of other classes that reference a class.
\end{itemize}

\end{frame}

%%%%%%%%%%%%%%%%%%%%%%%%%%%%%%%%%%%%%%%%%%

 \begin{frame}
\frametitle{Paper: Identifying logical dependencies from co-changing classes}

\end{frame}

%%%%%%%%%%%%%%%%%%%%%%%%%%%%%%%%%%%%%%%%%%

 \begin{frame}
\frametitle{Summary}

\end{frame}


\end{document}