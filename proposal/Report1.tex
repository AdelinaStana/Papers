
\documentclass[12pt]{mitthesis}
\usepackage{titlesec}
\usepackage{geometry}
\usepackage{graphicx}
\usepackage{booktabs, chemformula}
\usepackage{titlesec, blindtext, color}
\usepackage{listings}
\usepackage{float}
\usepackage{xcolor} 
\usepackage{array}
\usepackage{color, colortbl}
\usepackage{caption}
\usepackage{amsmath}

\captionsetup{font=footnotesize}

\definecolor{lightgreen}{rgb}{0.82, 0.94, 0.75}
\definecolor{lightorange}{rgb}{0.98, 0.84, 0.65}
\newcolumntype{L}{>{\centering\arraybackslash}m{10cm}}
\newcolumntype{l}{>{\centering\arraybackslash}m{7cm}}
\def\BibTeX{{\rm B\kern-.05em{\sc i\kern-.025em b}\kern-.08em
    T\kern-.1667em\lower.7ex\hbox{E}\kern-.125emX}}

\newcommand*{\Comb}[2]{{}^{#1}C_{#2}}%

\pagestyle{plain}

\begin{document}

\title{Methods and Tools for the Analysis of Legacy Software Systems}

\author{Stana Adelina Diana}


\department{Department of Computers and Information Technology}


\thesisdate{2021}


\supervisor{Vladimir Cretu}{Associate Professor}




\maketitle


\cleardoublepage

\setcounter{savepage}{\thepage}



\cleardoublepage



\tableofcontents

\pagestyle{plain}


\chapter{Introduction}


This report presents the results obtained so far on the proposed thesis. The goal of the thesis is to develop methods for analyzing legacy software systems, focusing on using historical information describing the evolution of the systems extracted from the versioning systems.

We have developed a tool that extracts and processes the needed information from a software system. The tool workflow and technologies used are presented in section \ref{sec:tool}. The primary information extracted by the tool is described in sections \ref{sec:sd_extraction} and \ref{sec:copairs_extraction}. The filtering methods selected to be applied to the extracted information are presented in sections \ref{sec:filtercommit}, \ref{sec:filterocc}, and \ref{sec:filterstrength}.

To perform measurements based on our assumptions, we have selected a set of 27 object-oriented software systems presented in section \ref{sec:dataset}. 
For each listed software system, the tool extracts, filters, and collects the information needed.

Each filtering section (\ref{sec:filtercommit}, \ref{sec:filterocc}, and \ref{sec:filterstrength}) contains the detailed results obtained after analyzing all the software systems and conclusions based on the results. 

Section \ref{sec:overlaps} focuses on the overlappings between the extracted information from the code and filtered information from the versioning systems. 

The conclusions and observations based on the performed measurements are presented in chapter \ref{conclusions}.




\chapter{Extracting software dependencies}
\label{extraction}


\section{Extracting structural dependencies}
\label{sec:sd_extraction}
A dependency is created between two elements that are in a relationship and indicates that an element of the relationship, in some manner, depends on the other element of the relationship \cite{Booch:2004:OAD:975416}, \cite{Cataldo2009SoftwareDW}.

Structural dependencies can be found by analyzing the source code \cite{Sangal:2005:UDM:1094811.1094824}, \cite{CalloArias2011}, \cite{structdep}. A structural dependency between two classes A and B is given by the fact that A statically depends on B, meaning that A cannot be compiled without knowing about B. In object oriented systems, this dependency can be given by many types of relationships between the two classes: A extends B, A implements B, A has attributes of type B, A has methods which have type B in their signature, A uses local variables of type B, A calls methods of B.


 We use an external tool called srcML \cite{srcML} to convert all source code files from the current release into XML files. All the information about classes, methods, calls to other classes are extracted by parsing the XML files and building a dependency data structure \cite{2003:XLC:851042.857028},
\cite{Collard:2011:LTF:2067850.2068011}. We choose the srcML format because it has the same markup for different programming languages and can ease the parsing of source code written in various programming languages such as Java, C++, and C\#.


\section{Extracting co-changing pairs}
\label{sec:copairs_extraction}

\textit{Logical dependencies} (a.k.a logical coupling) can be found by software history analysis and can reveal relationships that are not always present in the source code (structural dependencies).  

The concepts of logical coupling and logical dependencies were first used in different analysis tasks, all related to changes: for software change impact analysis \cite{1553643}, for identifying the potential ripple effects caused by software changes during software maintenance and evolution \cite{DBLP:conf/issre/OlivaG15}, \cite{Oliva:2011:ISL:2067853.2068086}, \cite{Poshyvanyk2009}, \cite{posh2010} or for their link to deffects \cite{wiese}, \cite{Zimmermann:2004:MVH:998675.999460}.

Software engineering practice has shown that sometimes modules which do not present structural dependencies still can be related \cite{articleEvolution}. Co-evolution represents the phenomenon when one component changes in response to a change in another component \cite{Yu:2007:UCC:1231330.1231370}, \cite{5166450}. Those changes can be found in the software change history from the versioning system. Gall \cite{Gall:1998:DLC:850947.853338}, \cite{Gall:2003:CRH:942803.943741} identified as logical coupling between two modules the fact that these modules \textit{repeatedly} change together during the historical evolution of the software system \cite{6606615}.


The versioning system contains the long-term change history of every file. Each project change made by an individual at a certain point of time is contained into a commit \cite{7471284}. All the commits are stored in the versioning system chronologically and each commit has a parent. The parent commit is the baseline from which development began, the only exception to this rule is the first commit which has no parent \cite{svn}. 


Currently there is no set of rules or best practices that can be applied to the extracted class co-changes and can guarantee their filtering into a set of logical dependencies.
This is mainly because not all the updates made in the versioning system are code related. For example a commit that has as participants a big number of files can indicate that a merge with another branch or a folder renaming has been made. In this case, a series of irrelevant co-changing pairs of entities can be introduced. So, in order to exclude this kind of situations the information extracted from the versioning system has to be filtered first and then used. Surveys also show that historical information is rarely used due to the size of the extracted information \cite{Shtern:2012:CMS:2332427.2332428}, \cite{sar}.

Other works have tried to filter co-changes \cite{Oliva:2011:ISL:2067853.2068086}, \cite{DBLP:journals/jss/AjienkaC17}, \cite{DBLP:conf/issre/OlivaG15}. One of the used co-changes filter is the commit size.The commit size is the number of code files changed in that particular commit. 
Ajienka and Capiluppi established a threshold of 10 for the maximum accepted size for a commit \cite{DBLP:journals/jss/AjienkaC17}. This means that all the commits that had more than 10 code files changed where discarded from the research. But setting a harcoded threshold for the commit size is debatable because in order to say that a commit is big or small you have to look first at the size of the system and at the trends from the versioning system. Even thought the best practices encourage small and often commits, the developers culture is the one that influences the most the trending size of commits from one system.

Filtering only after commit size is not enough, this type of filtering can indeed have an impact on the total number of extracted co-changes, but will only shrink the number of co-changes extracted without actually guaranteeing that the remaining ones have more relevancy and are more linked.

Although, some unrelated files can be updated by human error in small commits, for example: one file was forgot to be commited in the current commit and will be commited in the next one among some unrelated files. This kind of situation can introduce a set of co-changing pairs that are definitely not logical liked. In order to avoid this kind of situation a filter for the occurrence rate of co-changing pairs can be introduced. Co-changing pairs that occur multiple times are more prone to be logically dependent than the ones that occur only once. Currently there are no concrete examples of how the threshold for this type of filter can be calculated. In order to do that, incrementing the threshold by a certain step will be the start and then studying the impact on the remaining co-changing pairs for different systems. 

Nevertheless, logical dependencies should integrate harmoniously with structural dependencies in an unitary dependency model: valid logical dependencies should not be omitted from the dependency model, but structural dependencies should not be engulfed by questionable logical dependencies generated by casual co-changes.  Thus, in order to add logical dependencies besides structural dependencies in dependency models, class co-changes must be filtered until they remain only a reduced but relevant set of valid logical dependencies. 

\section{Tool for measuring software dependencies}
\label{sec:tool}

To establish structural and logical dependencies, we developed a tool that takes as input the source code repository URL of a given system and extracts from it the software dependencies \cite{DepSACI}. 
From a workflow point of view, we can identify 3 major types of activities that the tool does: downloads the required data from the git repository, extracts from the source code the structural dependencies and, extracts and filters the co-changing pairs from the repository's commit history. Figure \ref{fig:figworkflow} represents the activities mentioned above. Each block represents a different activity. 

\begin{figure}[H]
\centering
\includegraphics[width=\textwidth]{tool_workflow.png}
\caption{Tool workflow and major activities.}
\label{fig:figworkflow}
\end{figure}


\textbf{Download git data.}

The source code repository provides us all the needed information to extract both types of dependencies. It holds the code of the system but also the change history of the system. We use the source code for structural dependencies extraction \ref{sec:sd_extraction} and the change history for co-changing pairs extraction \ref{sec:copairs_extraction}.
To get the source code files and the change history, we first need to know the repository URL from GitHub (GitHub is a Git repository cloud-based hosting service). With the GitHub URL and a series of Git commands, the tool can download all the necessary data for dependencies extraction.

As we can see in figure \ref{fig:figgitdata}, the \textit{"clone"} command will download a Git repository to your local computer, including the source code files. The \textit{"diff"} command will get the differences between two existing commits in the Git repository. 
The tool gets the Git repository and the source code files by executing the "clone" command. Afterward, it gets all the existing commits within the Git repository. The commits are ordered by date, beginning with the oldest one and ending with the most recent one. The tool executes the "diff" command between each commit and its parent (the previous commit). The "diff" command generates a text file that contains the differences between the two commits: code differences, the number of files changed and changed file names.


\begin{figure}[H]
\centering
\includegraphics[width=\textwidth]{gitdata.png}
\caption{Commands used to download the required data from GitHub.}
\label{fig:figgitdata}
\end{figure}

\textbf{Extract structural dependencies.}

To extract the structural dependencies from the source code files the tool converts each source code file into srcML format using an open-source tool called srcML. The srcML format is an XML representation for source code. Each markup tag identifies elements of the abstract syntax for the language \cite{srcML}. 
After conversion, the tool parses each file and identifies all the defined entities (class, interface, enum, struct) within the file. It also identifies all the entities that are used by the entities defined.  The connection between both types of entities mentioned above constitutes a structural dependency.

\textbf{Extract and filter co-changing pairs.}

The process of extracting and filtering the co-changing pairs is represented in figure \ref{fig:figfiltering}.
For co-changing pairs extraction, the tool parses each generated diff file.
For each file, the tool gets the number of changed files and the name of the files. 
After structural dependencies extraction, the tool knows all the software entities contained in a file. Two entities from two changed files form a co-changing pair. After all the co-changing pairs of one diff file are extracted, the tool moves to the next diff file and extracts the set of co-changing pairs.

As presented in sections \ref{sec:filtercommit}, \ref{sec:filterocc}, and \ref{sec:filterstrength}, not every co-changing pair extracted is a logical dependency. For a co-changing pair to be labeled as a logical dependency, it has to meet some criteria. Each criterion constitutes a filter that a co-changing pair has to pass in order to be called logical dependency.
The filters are implemented in the tool and can be combined. The input for each filter is the set of co-changing pairs extracted, and the output is the remaining co-changing pairs that respect the filter criterion.


\begin{figure}[H]
\centering
\includegraphics[width=\textwidth]{pairs_filtering.png}
\caption{Co-changing pairs extraction and filtering.}
\label{fig:figfiltering}
\end{figure}


\chapter {Filtering extracted co-changing pairs in order to obtain logical dependencies}

\section{Data set used}
\label{sec:dataset}
We have analyzed a set of open-source projects found on GitHub\footnote{http://github.com/} \cite{Kalliamvakou2016} in order to extract the structural and logical dependencies between classes. Table \ref{table:1} enumerates all the systems studied. The 1st column assigns the projects IDs; 2nd column shows the project name; 3rd column shows the number of entities(classes and interfaces) extracted; 4th column shows the number of most recent commits analyzed from the active branch of each project and the 5th shows the language in which the project was developed.


\begin{table}[!h]
\renewcommand{\arraystretch}{1}
\caption{Summary of open source projects studied.}
\label{table:1}
\centering
\scalebox{0.9}{
\begin{tabular}{|c|c|c|c|c|c|}
\hline
   ID  & Project    & Nr. of & Nr. of& Type\\
     &     & entites & commits & \\
\hline
1	&	bluecove	&	2685	&	894	&	java	\\
2	&	aima-java	&	5232	&	1006	&	java	\\
3	&	powermock	&	2801	&	949	&	java	\\
4	&	restfb	&	3350	&	1391	&	java	\\
5	&	rxjava	&	21097	&	4398	&	java	\\
6	&	metro-jax-ws	&	6482	&	2927	&	java	\\
7	&	mockito	&	5189	&	3330	&	java	\\
8	&	grizzly	&	10687	&	3113	&	java	\\
9	&	shipkit	&	639	&	1563	&	java	\\
10	&	OpenClinica	&	9655	&	3276	&	java	\\
11	&	robolectric	&	8922	&	5912	&	java	\\
12	&	aeron	&	4159	&	5977	&	java	\\
13	&	antlr4	&	4747	&	4431	&	java	\\
14	&	mcidasv	&	3272	&	4136	&	java	\\
15	&	ShareX	&	4289	&	5485	&	csharp	\\
16	&	aspnetboilerplate	&	9712	&	4323	&	csharp	\\
17	&	orleans	&	16963	&	3995	&	csharp	\\
18	&	cli	&	2063	&	4488	&	csharp	\\
19	&	cake	&	12260	&	2518	&	csharp	\\
20	&	Avalonia	&	16732	&	5264	&	csharp	\\
21	&	EntityFrameworkCore	&	50179	&	5210	&	csharp	\\
22	&	jellyfin	&	8764	&	5433	&	csharp	\\
23	&	PowerShell	&	2405	&	3250	&	csharp	\\
24	&	WeiXinMPSDK	&	7075	&	5729	&	csharp	\\
25	&	ArchiSteamFarm	&	702	&	2497	&	csharp	\\
26	&	VisualStudio	&	4869	&	5039	&	csharp	\\
27	&	CppSharp	&	17060	&	4522	&	csharp	\\

\hline
\end{tabular}
}
\end{table}



\section{Filtering based on size of commit transactions}
\label{sec:filtercommit}

As presented in section \ref{sec:copairs_extraction}, according to surveys,  co-changing pairs are not used because of their size. One system can have millions of co-changing pairs.
Whit this filtering type, we not only want to decrease the total size of the extracted co-changing pairs. But also to be one step closer to the identification of the logical dependencies among the co-changing pairs.
In this step, we want to filter the co-changing pairs extracted after commit size (cs). This means that the co-changing pairs are extracted only from commits that involve fewer files than an established threshold number. 

Different works have chosen fixed threshold values for the maximum number of files accepted in a commit. Cappiluppi and Ajienka, in their works \cite{DBLP:journals/jss/AjienkaC17}, \cite{DBLP:journals/ese/AjienkaCC18} only take into consideration commits with less then 10 source code files changed in building the logical dependencies.

The research of Beck et al \cite{Beck:2011:CMC:2025113.2025162} only takes in consideration transactions with up to 25 files. The research \cite{Oliva:2011:ISL:2067853.2068086} provided also a quantitative analysis of the number of files per revision; Based on the analysis of 40,518 revisions, the mean value obtained for the number of files in a revision is 6 files. However, standard deviation value shows that the dispersion is high. 

We analyzed the overall transaction size trend for 27 open-source csharp and java systems with a total of 74 332 commits. The results are presented in Figure \ref{fig:fig_cs} and in table \ref{table:cs_values}, based on them we can say that 90\% of the total commit transactions made are with less than 10 source code files changed. This percent allows us to say that setting a threshold of 10 files for the maximum size of the commit transactions will not affect so much the total number of commit transactions from the systems since it will still remain 90\% of the commit transactions from where we can extract co-changing pairs \cite{DepSACI}.


\begin{figure}[!h]
\centering
\includegraphics[width=\textwidth]{commit_distribution.png}
\caption{Commit transaction size(cs) trend in percentages.}
\label{fig:fig_cs}
\centering
\end{figure}


\begin{figure}[!h]
\centering
\includegraphics[width=\textwidth]{ld_distribution.png}
\caption{Percentages of LD extracted from each commit transaction size(cs) group.}
\label{fig:fig_ld_cs}
\centering
\end{figure}

As we can see in Figure \ref{fig:fig_ld_cs} even though only 5\% of the commit transactions have more than 20 files changed ($20<cs<inf$) they generate in average 80\% of the total amount of co-changing pairs extracted from the systems.
The high number of co-changing pairs extracted from such a small number of commit transactions is caused by the number of files involved in those commit transactions. 

One single commit transaction can lead to a large amount of co-changing pairs. For example in RxJava we have commit transactions with 1030 source code files, this means that those commits can generate 
$\Comb{n}{k}=\frac{n!}{k!(n-k)!} = \frac{1030!}{2!(1028)!} = 529 935$ logical dependencies. By setting a threshold on the commit transaction size we can avoid the introduction of those co-changing pairs into the system.

So filtering 10\% of the total amount of commit transactions can lead to a significant decrease of the amount of co-changing pairs and that is why we choose the value of 10 files as our fixed threshold for the maximum size of a commit transaction \cite{DepSACI}.



\begin{table}[!h]
\renewcommand{\arraystretch}{1}
\caption{Commit transaction size(cs) trend and average per system.}
\label{table:cs_values}
\centering
\scalebox{0.9}{
\begin{tabular}{|c|c|c|c|c|c|c|}
\hline
Nr.	  & Project    &	$cs\leq 5$	&	$cs\leq 10$	&	$cs\leq 20$	&	$cs<\infty$ & Avg	\\ 
\hline
1	&	bluecove	&	738	&	97	&	37	&	22	&	4.9	\\
2	&	aima-java	&	733	&	134	&	74	&	65	&	7.24	\\
3	&	powermock	&	685	&	128	&	66	&	70	&	9.61	\\
4	&	restfb	&	1160	&	127	&	44	&	60	&	9.9	\\
5	&	rxjava	&	3395	&	447	&	253	&	303	&	8.46	\\
6	&	metro-jax-ws	&	2583	&	198	&	78	&	68	&	4.33	\\
7	&	mockito	&	2522	&	433	&	222	&	153	&	6.33	\\
8	&	grizzly	&	2487	&	302	&	180	&	144	&	5.28	\\
9	&	shipkit	&	1311	&	151	&	64	&	37	&	4.26	\\
10	&	OpenClinica	&	2837	&	250	&	119	&	70	&	3.31	\\
11	&	robolectric	&	4827	&	503	&	264	&	318	&	7.43	\\
12	&	aeron	&	4844	&	684	&	300	&	149	&	4.6	\\
13	&	antlr4	&	3426	&	437	&	304	&	264	&	8.5	\\
14	&	mcidasv	&	3996	&	81	&	35	&	24	&	2.47	\\
15	&	ShareX	&	4731	&	529	&	145	&	80	&	4.69	\\
16	&	aspnetboilerplate	&	3208	&	569	&	321	&	225	&	6.61	\\
17	&	orleans	&	2780	&	518	&	369	&	328	&	8.95	\\
18	&	cli	&	3377	&	551	&	308	&	252	&	6.43	\\
19	&	cake	&	1785	&	359	&	174	&	200	&	9.89	\\
20	&	Avalonia	&	3806	&	641	&	371	&	446	&	8.43	\\
21	&	EntityFrameworkCore	&	2866	&	878	&	644	&	822	&	15.38	\\
22	&	jellyfin	&	4007	&	662	&	419	&	345	&	6.25	\\
23	&	PowerShell	&	2702	&	224	&	133	&	191	&	7.33	\\
24	&	WeiXinMPSDK	&	4604	&	526	&	296	&	303	&	9.01	\\
25	&	ArchiSteamFarm	&	2357	&	92	&	28	&	20	&	2.24	\\
26	&	VisualStudio	&	3902	&	521	&	295	&	321	&	6.71	\\
27	&	CppSharp	&	3870	&	390	&	203	&	59	&	3.28	\\
\hline
\end{tabular}
}
\end{table}





\section{Filtering based on number of occurrences}
\label{sec:filterocc}

In the previous section, we filtered the co-changing pairs based on the commit size. Even though the number of extracted co-changing pairs was reduced, this type of filtering will not guarantee that the remaining co-changing pairs can pass as logical dependencies. 
One occurrence of a co-change pair can be a valid logical dependency, but can also be a coincidence. 

Taking into consideration only co-changing pairs with multiple occurrences as valid dependencies can lead to more accurate results. But, if the project studied has a relatively small amount of commits, the probability to find multiple updates of the same classes at the same time is less likely to happen, so filtering after the number of occurrences can lead to filtering all the co-changes extracted.

We have performed a series of analyses on the test systems, incrementing the threshold value occurrence (occ) from 1 to 4. The co-changing pairs are extracted only for commits with the commit transaction size less or equal to 10. For each threshold mentioned above, the extracted co-changing pairs are filtered again by the occurrence threshold established. All the co-changing pairs that do not exceed the minimum number of occurrences are discarded.

The results of the analysis are presented in Table \ref{table:sd_percentages} as percentages of co-changing pairs that are also structural dependencies and Table \ref{table:ld_ratio} as ratio of the number of co-changing pairs to the number of structural dependencies (SD).


\begin{table}[!h]
\renewcommand{\arraystretch}{1}
\caption{Percentage of co-changing pairs that are also structural dependencies.}
\label{table:sd_percentages}
\centering
\scalebox{0.9}{
\begin{tabular}{|c|c|c|c|c|}
\hline
    ID  & $occ\geq 1$ & $occ\geq 2$ & $occ\geq 3$ & $occ\geq 4$  \\
\hline
1	&	7,13	&	7,77	&	7,99	&	19,71	\\
2	&	19,54	&	25,76	&	29,55	&	32,16	\\
3	&	6,66	&	8,58	&	11,82	&	14,87	\\
4	&	1,16	&	1,17	&	0,91	&	0,80	\\
5	&	3,99	&	3,96	&	7,75	&	7,49	\\
6	&	13,92	&	20,16	&	22,91	&	22,77	\\
7	&	8,38	&	9,28	&	14,93	&	14,58	\\
8	&	6,70	&	9,73	&	14,20	&	15,60	\\
9	&	16,98	&	23,34	&	29,22	&	32,89	\\
10	&	8,94	&	9,15	&	11,05	&	10,59	\\
11	&	4,99	&	6,92	&	8,88	&	11,08	\\
12	&	13,19	&	17,15	&	18,60	&	19,57	\\
13	&	2,43	&	5,59	&	8,33	&	8,21	\\
14	&	13,27	&	18,88	&	19,02	&	19,28	\\
15	&	12,90	&	21,95	&	25,51	&	27,01	\\
16	&	13,33	&	17,34	&	18,53	&	16,24	\\
17	&	6,09	&	6,18	&	6,41	&	6,44	\\
18	&	9,73	&	10,60	&	14,27	&	18,80	\\
19	&	10,26	&	13,54	&	13,64	&	12,60	\\
20	&	12,83	&	18,36	&	21,00	&	25,72	\\
21	&	2,86	&	4,65	&	5,70	&	4,98	\\
22	&	5,20	&	6,56	&	8,18	&	8,90	\\
23	&	8,23	&	13,64	&	17,04	&	17,65	\\
24	&	6,77	&	10,89	&	14,47	&	16,05	\\
25	&	9,85	&	10,15	&	11,65	&	11,33	\\
26	&	8,65	&	10,79	&	12,78	&	14,34	\\
27	&	7,04	&	8,78	&	9,87	&	10,08	\\
\hline
Avg	&	8,93	&	11,88	&	14,23	&	15,55	\\
\hline
\end{tabular}
}
\end{table}


\begin{table}[!h]
\renewcommand{\arraystretch}{1}
\caption{Ratio of number of co-changing pairs to number of structural dependencies. }
\label{table:ld_ratio}
\centering
\scalebox{0.9}{
\begin{tabular}{|c|c|c|c|c|}
\hline
    ID  & $occ\geq 1$ & $occ\geq 2$ & $occ\geq 3$ & $occ\geq 4$  \\
\hline
1	&	4,13	&	1,94	&	1,23	&	0,26	\\
2	&	0,81	&	0,33	&	0,16	&	0,10	\\
3	&	5,12	&	1,93	&	0,78	&	0,38	\\
4	&	53,36	&	42,00	&	38,31	&	36,30	\\
5	&	4,27	&	2,90	&	0,88	&	0,72	\\
6	&	1,07	&	0,46	&	0,30	&	0,23	\\
7	&	4,09	&	2,38	&	0,99	&	0,73	\\
8	&	4,06	&	1,57	&	0,76	&	0,49	\\
9	&	3,64	&	2,03	&	1,14	&	0,77	\\
10	&	1,41	&	1,01	&	0,47	&	0,34	\\
11	&	7,91	&	4,47	&	2,93	&	2,03	\\
12	&	3,92	&	2,15	&	1,47	&	1,07	\\
13	&	10,15	&	3,18	&	1,22	&	1,03	\\
14	&	3,07	&	1,53	&	1,16	&	0,97	\\
15	&	2,34	&	0,84	&	0,48	&	0,33	\\
16	&	1,21	&	0,47	&	0,26	&	0,19	\\
17	&	2,99	&	1,83	&	1,11	&	0,84	\\
18	&	2,26	&	1,37	&	0,67	&	0,40	\\
19	&	2,32	&	1,38	&	0,76	&	0,67	\\
20	&	1,24	&	0,58	&	0,35	&	0,18	\\
21	&	5,33	&	2,12	&	1,27	&	1,05	\\
22	&	3,38	&	1,88	&	0,99	&	0,74	\\
23	&	3,62	&	1,22	&	0,76	&	0,37	\\
24	&	2,57	&	1,22	&	0,67	&	0,46	\\
25	&	7,47	&	5,36	&	4,16	&	3,73	\\
26	&	4,03	&	2,16	&	1,50	&	1,15	\\
27	&	7,46	&	4,26	&	2,99	&	2,43	\\
\hline
Avg	&	5,67	&	3,43	&	2,51	&	2,15	\\
\hline
\end{tabular}
}
\end{table}

Based on Table \ref{table:sd_percentages} we can say that only a small percentage of the extracted co-changing pairs are also structural dependencies. This is consistent with the findings of related works \cite{DBLP:journals/jss/AjienkaC17}, \cite{DBLP:journals/ese/AjienkaCC18}. 
The percentage of co-changing pairs that are also structural dependencies increases with the minimum number of occurrences because the number of co-changing pairs from the systems decreases with the minimum number of occurrences. 
We calculate the overlapping between co-changing pairs and structural dependencies not only because we want to get an idea of how many structural dependencies are reflected in the versioning system through co-changing pairs, but also because we want to eliminate co-changing pairs that are structural dependencies since they don't bring any new information about the system.

We stopped the minimum occurrences threshold to 4 because we observed that for systems with ID 2, 6, 10, and 16 from Table \ref{table:ld_ratio} the ratio number is lower than 1, which means that the number of structural dependencies is higher than the number of co-changing pairs. On the other hand, for systems with ID 4, 11, 25, 27, the threshold of 4 for a minimum number of occurrences does not change the discrepancy between the number of co-changing pairs and structural dependencies.

If we try to go higher with the occurrences threshold, we will risk filtering all the existing co-changing pairs for some systems.
So, filtering with a threshold of 4 for the minimum number of occurrences will indeed filter the logical dependencies, but for some of the systems, the remaining number of co-changing pairs will still be significantly higher compared to the number of structural dependencies.




\section{Filtering based on connection strenght}
\label{sec:filterstrength}

In section \ref{sec:filtercommit} we filtered the co-changing pairs extracted from the versioning system history based on the commit size. Based on the results obtained, we decided to filter out all co-changing pairs extracted from commits with more than 10 files changed. 

In section \ref{sec:filterocc}, we added a new filtering rule based on the occurrence of a co-changing pair. The new filter is applied to the co-changing pairs resulted after commit size filtering. In this case, the filtering method proved insufficient due to the size diversity of the systems. One important conclusion drawn from the occurrence number filtering is that setting a hard threshold for a filter is not always a good idea. One threshold value can be too much for a small-sized system and too little for a medium-sized system. 

To avoid the above problem, we decided to introduce another filter complementary to the commit size filter described in section \ref{sec:filtercommit}.
This filter focuses on the connection strength of a co-changing pair. In this section, we will filter out all the co-changing pairs that are not strongly connected.

To determine the connection strength of a pair, we first need to calculate the connection factors for both entities that form a co-changing pair.
Assuming that we have a co-changing pair formed by entities A and B, the connection factor of entity A with entity B is the percentage from the total commits involving A that contains entity B. The connection factor of entity B with entity A is the percentage from the total commits involving B that contain also entity A.

\[ connection\ factor\ for\ A 
  = \dfrac{100 * commits\ involving\ A\ and\ B}{total\ nr\ of\ commits\ involving\ A}
\]

\[ connection\ factor\ for\ B 
  = \dfrac{100 * commits\ involving\ A\ and\ B}{total\ nr\ of\ commits\ involving\ B}
\]

As a practical example, if the pair formed by A and B update together 7 times and the total number of commits involving A is 20 and involving B is 7. The factor for A is 35 and for B is 100. The factor of 100 is the maximum factor that you can have and means that in all the commits involving B, also A is present.

Due to the fact that the factors obtained can vary from 0 to 100, for this filter, we begin with a threshold value of 10 and increment it by 10 until we reach 100. 

The co-changing pairs are filtered out based on two scenarios:
\begin{itemize}
	\item factor A and factor B $\geq threshold \%$ 
	\item factor A or factor B $\geq threshold \%$ 
\end{itemize}

\begin{table}[!h]
\renewcommand{\arraystretch}{1}
\caption{Ratio of number of filtered co-changing pairs to number of SD, when factor A and factor B $\geq threshold \%$ }
\label{tab:commitstrengthAND}
\centering
\scalebox{0.8}{
\begin{tabular}{|c|cccccccccc|c|}
\hline
Project &	$\geq10\%$	&	$\geq20\%$		&	$\geq30\%$		&	$\geq40\%$		&	$\geq50\%$		&	$\geq60\%$		&	$\geq70\%$		&	$\geq80\%$		&	$\geq90\%$		&	$\geq100\%$	 \\
\hline
bluecove	&	1.326	&	0.658	&	0.433	&	0.401	&	0.244	&	0.199	&	0.195	&	0.022	&	0.011	&	0.011	\\
aima-java	&	0.266	&	0.137	&	0.070	&	0.044	&	0.036	&	0.019	&	0.005	&	0.004	&	0.003	&	0.003	\\
powermock	&	0.505	&	0.243	&	0.147	&	0.086	&	0.061	&	0.031	&	0.031	&	0.031	&	0.031	&	0.031	\\
restfb	&	0.822	&	0.163	&	0.045	&	0.017	&	0.011	&	0.002	&	0.001	&	0.001	&	0.001	&	0.001	\\
rxjava	&	0.234	&	0.119	&	0.054	&	0.037	&	0.034	&	0.018	&	0.013	&	0.011	&	0.007	&	0.007	\\
metro-jax-ws	&	0.227	&	0.155	&	0.101	&	0.077	&	0.070	&	0.036	&	0.018	&	0.017	&	0.016	&	0.016	\\
mockito	&	1.590	&	0.804	&	0.357	&	0.288	&	0.215	&	0.088	&	0.052	&	0.036	&	0.032	&	0.032	\\
grizzly	&	2.073	&	0.293	&	0.170	&	0.111	&	0.093	&	0.050	&	0.039	&	0.034	&	0.021	&	0.007	\\
shipkit	&	1.495	&	0.479	&	0.271	&	0.142	&	0.108	&	0.059	&	0.047	&	0.011	&	0.008	&	0.008	\\
OpenClinica	&	0.253	&	0.135	&	0.093	&	0.078	&	0.062	&	0.042	&	0.024	&	0.019	&	0.019	&	0.017	\\
robolectric	&	0.114	&	0.086	&	0.064	&	0.037	&	0.027	&	0.025	&	0.001	&	0.000	&	0.000	&	0.000	\\
aeron	&	0.277	&	0.136	&	0.085	&	0.069	&	0.053	&	0.045	&	0.039	&	0.015	&	0.007	&	0.004	\\
antlr4	&	11.363	&	0.721	&	0.031	&	0.010	&	0.007	&	0.004	&	0.000	&	0.000	&	0.000	&	0.000	\\
mcidasv	&	3.225	&	0.805	&	0.660	&	0.533	&	0.493	&	0.454	&	0.386	&	0.356	&	0.005	&	0.005	\\
ShareX	&	6.097	&	0.725	&	0.663	&	0.564	&	0.500	&	0.242	&	0.176	&	0.170	&	0.001	&	0.001	\\
aspnetboilerplate	&	1.302	&	0.333	&	0.219	&	0.146	&	0.094	&	0.045	&	0.014	&	0.008	&	0.007	&	0.007	\\
orleans	&	0.816	&	0.640	&	0.551	&	0.503	&	0.496	&	0.196	&	0.159	&	0.152	&	0.142	&	0.142	\\
cli	&	1.676	&	0.233	&	0.159	&	0.118	&	0.102	&	0.062	&	0.058	&	0.029	&	0.026	&	0.026	\\
cake	&	2.335	&	0.753	&	0.614	&	0.337	&	0.075	&	0.021	&	0.007	&	0.004	&	0.004	&	0.004	\\
Avalonia	&	0.846	&	0.117	&	0.098	&	0.018	&	0.013	&	0.002	&	0.001	&	0.001	&	0.001	&	0.001	\\
EntityFrameworkCore	&	3.377	&	1.691	&	1.608	&	1.584	&	1.576	&	1.310	&	0.001	&	0.001	&	0.001	&	0.001	\\
jellyfin	&	0.132	&	0.006	&	0.003	&	0.002	&	0.002	&	0.000	&	0.000	&	0.000	&	0.000	&	0.000	\\
PowerShell	&	1.732	&	1.299	&	0.158	&	0.053	&	0.007	&	0.001	&	0.000	&	0.000	&	0.000	&	0.000	\\
WeiXinMPSDK	&	3.295	&	0.334	&	0.188	&	0.061	&	0.017	&	0.006	&	0.003	&	0.001	&	0.000	&	0.000	\\
ArchiSteamFarm	&	0.897	&	0.479	&	0.429	&	0.423	&	0.412	&	0.403	&	0.339	&	0.009	&	0.001	&	0.000	\\
VisualStudio	&	1.281	&	0.090	&	0.053	&	0.028	&	0.020	&	0.013	&	0.006	&	0.001	&	0.001	&	0.001	\\
CppSharp	&	99.528	&	1.020	&	0.992	&	0.980	&	0.972	&	0.927	&	0.078	&	0.075	&	0.073	&	0.072	\\

\hline
\end{tabular}
}
\end{table}


\begin{table}[!h]
\renewcommand{\arraystretch}{1}
\caption{Ratio of number of filtered co-changing pairs to number of SD,
 when factor A or factor B $\geq threshold \%$ }
\label{tab:commitstrengthOR}
\centering
\scalebox{0.8}{
\begin{tabular}{|c|cccccccccc|c|}
\hline
Project &	$\geq10\%$	&	$\geq20\%$		&	$\geq30\%$		&	$\geq40\%$		&	$\geq50\%$		&	$\geq60\%$		&	$\geq70\%$		&	$\geq80\%$		&	$\geq90\%$		&	$\geq100\%$	 \\
\hline
bluecove	&	1.312	&	1.181	&	0.700	&	0.599	&	0.419	&	0.235	&	0.219	&	0.046	&	0.045	&	0.045	\\
aima-java	&	0.430	&	0.280	&	0.176	&	0.118	&	0.103	&	0.056	&	0.022	&	0.020	&	0.020	&	0.020	\\
powermock	&	0.508	&	0.328	&	0.234	&	0.179	&	0.150	&	0.092	&	0.091	&	0.091	&	0.091	&	0.091	\\
restfb	&	0.662	&	0.336	&	0.122	&	0.067	&	0.059	&	0.016	&	0.015	&	0.015	&	0.015	&	0.015	\\
rxjava	&	0.279	&	0.206	&	0.145	&	0.100	&	0.099	&	0.047	&	0.044	&	0.039	&	0.034	&	0.034	\\
metro-jax-ws	&	0.271	&	0.261	&	0.204	&	0.172	&	0.160	&	0.106	&	0.082	&	0.081	&	0.080	&	0.080	\\
mockito	&	2.481	&	1.521	&	0.904	&	0.623	&	0.411	&	0.199	&	0.128	&	0.107	&	0.101	&	0.101	\\
grizzly	&	1.332	&	0.838	&	0.515	&	0.320	&	0.288	&	0.142	&	0.117	&	0.106	&	0.090	&	0.076	\\
shipkit	&	1.376	&	1.083	&	0.725	&	0.515	&	0.424	&	0.191	&	0.149	&	0.105	&	0.094	&	0.094	\\
OpenClinica	&	0.830	&	0.434	&	0.314	&	0.256	&	0.217	&	0.130	&	0.093	&	0.082	&	0.080	&	0.072	\\
robolectric	&	0.366	&	0.122	&	0.088	&	0.046	&	0.031	&	0.027	&	0.003	&	0.002	&	0.002	&	0.002	\\
aeron	&	0.781	&	0.449	&	0.265	&	0.190	&	0.160	&	0.096	&	0.062	&	0.031	&	0.021	&	0.018	\\
antlr4	&	11.363	&	0.798	&	0.055	&	0.022	&	0.011	&	0.007	&	0.002	&	0.002	&	0.002	&	0.002	\\
mcidasv	&	1.932	&	1.203	&	0.858	&	0.682	&	0.579	&	0.473	&	0.396	&	0.365	&	0.013	&	0.013	\\
ShareX	&	2.681	&	1.292	&	0.916	&	0.730	&	0.593	&	0.287	&	0.210	&	0.201	&	0.017	&	0.017	\\
aspnetboilerplate	&	1.055	&	0.759	&	0.493	&	0.364	&	0.273	&	0.130	&	0.067	&	0.050	&	0.046	&	0.046	\\
orleans	&	1.120	&	0.962	&	0.849	&	0.750	&	0.744	&	0.559	&	0.482	&	0.476	&	0.466	&	0.466	\\
cli	&	1.676	&	0.762	&	0.560	&	0.434	&	0.375	&	0.269	&	0.237	&	0.149	&	0.142	&	0.142	\\
cake	&	1.883	&	1.197	&	1.001	&	0.541	&	0.185	&	0.103	&	0.019	&	0.013	&	0.013	&	0.013	\\
Avalonia	&	0.510	&	0.224	&	0.138	&	0.037	&	0.028	&	0.011	&	0.006	&	0.003	&	0.003	&	0.003	\\
EntityFrameworkCore	&	2.636	&	1.888	&	1.695	&	1.623	&	1.608	&	1.317	&	0.006	&	0.006	&	0.006	&	0.006	\\
jellyfin	&	0.132	&	0.030	&	0.016	&	0.011	&	0.008	&	0.003	&	0.002	&	0.002	&	0.002	&	0.002	\\
PowerShell	&	3.454	&	1.648	&	0.232	&	0.081	&	0.021	&	0.004	&	0.003	&	0.003	&	0.003	&	0.003	\\
WeiXinMPSDK	&	1.342	&	0.603	&	0.327	&	0.144	&	0.080	&	0.047	&	0.015	&	0.008	&	0.007	&	0.007	\\
ArchiSteamFarm	&	5.472	&	1.416	&	0.830	&	0.677	&	0.575	&	0.450	&	0.353	&	0.023	&	0.016	&	0.014	\\
VisualStudio	&	1.281	&	0.236	&	0.142	&	0.092	&	0.060	&	0.040	&	0.031	&	0.020	&	0.019	&	0.019	\\
CppSharp	&	55.038	&	1.343	&	1.106	&	1.044	&	1.030	&	0.983	&	0.449	&	0.443	&	0.441	&	0.439	\\


\hline
\end{tabular}
}
\end{table}


In table \ref{tab:commitstrengthAND} we have on the columns the ratio between the number of structural dependencies and the number of co-changing pairs that resulted after filtering out pairs that have at least one factor below the specified threshold in the column header.
In table \ref{tab:commitstrengthOR} we have on the columns the ratio between the number of structural dependencies and the number of co-changing pairs that resulted after filtering out pairs that have both factors below the specified threshold in the column header.

We calculate the ratio number between the co-changing pairs and the structural dependencies because we want to evaluate the size of the extracted co-changing pairs compared to the size of the structural dependencies from the system. 
According to surveys \cite{Shtern:2012:CMS:2332427.2332428}, \cite{sar}, the main reason why logical dependencies (a.k.a filtered co-changes) are not used together with structural dependencies is because of their size. So, it is important to us to get at each filtering step an overview regarding the ratio between co-changes size and structural dependencies size.

From the results presented in tables \ref{tab:commitstrengthAND} and \ref{tab:commitstrengthOR} we conclude that the number of co-changing pairs is drastically reduced. In most cases, the number of structural dependencies surpasses the number of co-changing pairs that remain after filtering. But, we do the filtering not only to reduce the size of the co-changing pairs extracted. We do the filtering of co-changing pairs extracted to make sure that the remaining co-changing pairs are indeed logically dependent.

If we filter out all the co-changing pairs that do not update at least half of the time together (actor A and factor B $\geq threshold \%$ ) we remain with a decent quantity of co-changing pairs. Given the size of the output and the connection strength of the co-changing pairs, the remaining co-changing pairs can be considered, at this point, to be logically dependent. 




\section{Overlaps between structural and co-changing pairs}
\label{sec:overlaps}

A logical dependency can be also a structural dependency and vice-versa, so studying the overlapping between logical and structural dependencies while filtering is important since the intention is to introduce those logical dependencies among with structural dependencies in architectural reconstruction systems. Current studies have shown a relatively small percentage of overlapping between them with and without any kind of filtering \cite{DBLP:journals/jss/AjienkaC17}. This means that a lot of non related entities update together in the versioning system, the goal here is to establish the factors that determine such a small percentage of overlapping \cite{enase19}.

Since we are first extracting co-changing pairs and only after various filters we call the remaining co-changing pairs logically dependent, we will be studying the overlapping between the remaining co-changing pairs after each filtering stage and the structural dependencies. 
For each system, we extracted the structural dependencies and the co-changing pairs and determined the overlap between the two dependencies sets, in various experimental conditions. 

One variable experimental condition is whether changes located in comments contribute towards logical dependencies. This condition distinguishes between two different cases: 
\begin{itemize}
	\item with comments: a change in source code files is counted as a co-changing pair, even if the change is inside comments in all files
	\item without comments: commits that changed source code files only by editing comments are ignored
\end{itemize}

In all cases, we varied the following threshold values: 
 \begin{itemize}
	\item commit size ($cs$): the maximum size of commit transactions which are accepted to generate co-changes. The values for this threshold were 5, 10, 20 and no threshold (infinity).  
	\item number of occurrences ($occ$): the minimum number of repeated occurrences for a co-change to be counted as logical dependency. The values for this threshold were 1, 2, 3 and 4.  
\end{itemize}

The six tables below present the synthesis of our experiments. 
We have computed the following  values:
\begin{itemize}
	\item the mean ratio of the number of co-changes to the number of structural dependencies (SD)
   	\item the mean percentage of structural dependencies that are also co-changes (calculated from the number of overlaps divided to the number of structural dependencies)	
	\item the mean percentage of co-changes that are also structural dependencies (calculated from the number of overlaps divided to the number of co-changes)
\end{itemize}

In all the six tables, \ref{tab:ratio:comm}, \ref{tab:ratio:nocomm}, \ref{tab:percSD:comm}, \ref{tab:percSD:nocomm},
\ref{tab:percLD:comm}, \ref{tab:percLD:nocomm} we have on columns the values used for the commit size $cs$, while on rows we have the values for the number of occurrences threshold $occ$. The tables contain median values obtained for experiments done under all combinations of the two threshold values, on all test systems. In all tables, the upper right corner corresponds to the most relaxed filtering conditions, while the lower left corner corresponds to the most restrictive filtering conditions.

\begin{table}[!h]
%% increase table row spacing, adjust to taste
\renewcommand{\arraystretch}{1}
\caption{Ratio of number of co-changes to number of SD, case with comments}
\label{tab:ratio:comm}
\centering

\begin{tabular}{|c|c|c|c|c|}
\hline
	      &	$cs\leq 5$	&	$cs\leq 10$	&	$cs\leq 20$	&	$cs<\infty$	\\
\hline
$occ\geq 1$	&	3,39	&	5,67	&	9,00	&	80,31	\\
$occ\geq 2$	&	2,24	&	3,47	&	5,02	&	60,14	\\
$occ\geq 3$	&	1,04	&	2,53	&	3,52	&	44,68	\\
$occ\geq 4$	&	0,90	&	2,16	&	2,88	&	33,47	\\
\hline
\end{tabular}
\end{table}

\begin{table}[!h]
%% increase table row spacing, adjust to taste
\renewcommand{\arraystretch}{1}
\caption{Ratio of number of co-changes to number of SD, case without comments}
\label{tab:ratio:nocomm}
\centering

\begin{tabular}{|c|c|c|c|c|}
\hline
	      &	$cs\leq 5$	&	$cs\leq 10$	&	$cs\leq 20$	&	$cs< \infty$	\\
\hline
$occ\geq 1$	&	3,24	&	5,33	&	7,90	&	67,16	\\
$occ\geq 2$	&	1,35	&	3,27	&	4,72	&	47,39	\\
$occ\geq 3$	&	1,00	&	1,67	&	2,49	&	32,39	\\
$occ\geq 4$	&	0,43	&	1,26	&	1,93	&	22,15	\\
\hline
\end{tabular}
\end{table}

\begin{table}[!h]
%% increase table row spacing, adjust to taste
\renewcommand{\arraystretch}{1}
\caption{Percentage of SD that are also co-changes, case with comments}
\label{tab:percSD:comm}
\centering

\begin{tabular}{|c|c|c|c|c|}
\hline
	      &	$cs\leq 5$	&	$cs\leq 10$	&	$cs\leq 20$	&	$cs< \infty$	\\
\hline
$occ\geq 1$	&	19,75	&	29,86	&	39,29	&	76,59	\\
$occ\geq 2$	&	12,50	&	20,20	&	27,68	&	66,11	\\
$occ\geq 3$	&	8,49	&	14,22	&	19,94	&	55,99	\\
$occ\geq 4$	&	6,58	&	10,95	&	15,76	&	47,12	\\
\hline
\end{tabular}
\end{table}

\begin{table}[!h]
%% increase table row spacing, adjust to taste
\renewcommand{\arraystretch}{1}
\caption{Percentage of SD that are also co-changes, case without comments}
\label{tab:percSD:nocomm}
\centering

\begin{tabular}{|c|c|c|c|c|}
\hline
	      &	$cs\leq 5$	&	$cs\leq 10$	&	$cs\leq 20$	&	$cs< \infty$	\\
\hline
$occ\geq 1$	&	18,88	&	28,47	&	37,44	&	71,12	\\
$occ\geq 2$	&	11,87	&	19,03	&	25,93	&	59,58	\\
$occ\geq 3$	&	8,00	&	13,09	&	18,15	&	48,65	\\
$occ\geq 4$	&	5,85	&	9,94	&	14,27	&	39,07	\\
\hline
\end{tabular}
\end{table}

\begin{table}[!h]
%% increase table row spacing, adjust to taste
\renewcommand{\arraystretch}{1}
\caption{Percentage of co-changes that are also SD, case with comments}
\label{tab:percLD:comm}
\centering

\begin{tabular}{|c|c|c|c|c|}
\hline
	      &	$cs\leq 5$	&	$cs\leq 10$	&	$cs\leq 20$	&	$cs< \infty$	\\
\hline
$occ\geq 1$	&	12,02	&	8,86	&	6,72	&	1,79	\\
$occ\geq 2$	&	15,05	&	11,71	&	9,38	&	2,21	\\
$occ\geq 3$	&	17,45	&	13,97	&	11,57	&	2,86	\\
$occ\geq 4$	&	18,96	&	15,28	&	12,94	&	3,67	\\
\hline
\end{tabular}
\end{table}

\begin{table}[!h]
%% increase table row spacing, adjust to taste
\renewcommand{\arraystretch}{1}
\caption{Percentage of co-changes that are also SD, case without comments}
\label{tab:percLD:nocomm}
\centering
\begin{tabular}{|c|c|c|c|c|}
\hline
	      &	$cs\leq 5$	&	$cs\leq 10$	&	$cs\leq 20$	&	$cs< \infty$	\\
\hline
$occ\geq 1$	&	12,05	&	9,02	&	6,98	&	1,93	\\
$occ\geq 2$	&	15,08	&	12,03	&	9,66	&	2,42	\\
$occ\geq 3$	&	17,78	&	14,37	&	12,24	&	3,28	\\
$occ\geq 4$	&	19,22	&	15,59	&	13,30	&	4,21	\\
\hline
\end{tabular}
\end{table}



\begin{table}[!h]
\renewcommand{\arraystretch}{1}
\caption{Percentage of SD that are also co-changing pairs after connection strength filtering. }
\label{tab:percSDstrength}
\centering
\scalebox{0.8}{
\begin{tabular}{|c|cccccccccc|c|}
\hline
Condition &	$\geq10\%$	&	$\geq20\%$		&	$\geq30\%$		&	$\geq40\%$		&	$\geq50\%$		&	$\geq60\%$		&	$\geq70\%$		&	$\geq80\%$		&	$\geq90\%$		&	$\geq100\%$	 \\
\hline
							
factor A and factor B	&	11.20	&	6.80		&	4.44	&	3.25	&	2.58	&	1.74		&	1.16	&	0.57	&	0.35	&	0.33	\\
factor A or factor B	&	15.94	&	11.02	&	7.56	&	5.59		&	4.52	&	2.90	&	2.00	&	1.33	&	1.04	&	1.02	\\
								
\hline
\end{tabular}
}
\end{table}

\begin{table}[!h]
\renewcommand{\arraystretch}{1}
\caption{Percentage of co-changing pairs that are SD after connection strength filtering. }
\label{tab:percLDtrength}
\centering
\scalebox{0.8}{
\begin{tabular}{|c|cccccccccc|c|}
\hline
Condition &	$\geq10\%$	&	$\geq20\%$		&	$\geq30\%$		&	$\geq40\%$		&	$\geq50\%$		&	$\geq60\%$		&	$\geq70\%$		&	$\geq80\%$		&	$\geq90\%$		&	$\geq100\%$	 \\
\hline
factor A and factor B	&	10.95	&	20.61	&	23.73	&	26.75	&	28.57	&	33.31	&	33.43	&	38.34	&	42.52	&	39.41	\\
factor A or factor B		&	12.19	&   16.85	&	19.41	&	20.70	&	21.63	&	22.84	&	21.86	&	23.08	&	24.00	&	22.73	\\						

\hline
\end{tabular}
}
\end{table}

In order to assess the influence of comments, we compare pairwise Tables \ref{tab:ratio:comm} and \ref{tab:ratio:nocomm},  
Tables \ref{tab:percSD:comm} and \ref{tab:percSD:nocomm} and Tables \ref{tab:percLD:comm} and \ref{tab:percLD:nocomm}. 
We observe that, although there are some differences between pairs of measurements done in similar conditions with and without comments, the differences are not significant.

On the other hand, the overlap between structural and co-changes is given by the number of pairs of classes that have both structural and co-change dependencies. We evaluate this overlap as a percentage relative to the number of structural dependencies in Tables \ref{tab:percSD:comm},\ref{tab:percSD:nocomm} and \ref{tab:percSDstrength}, respectively as a percentage relative to the number of co-changes in Tables \ref{tab:percLD:comm},\ref{tab:percLD:nocomm}, \ref{tab:percLDtrength}.

A first observation from Tables \ref{tab:percSD:comm}, \ref{tab:percSD:nocomm}, and \ref{tab:percSDstrength} is that not all pairs of classes with structural dependencies co-change. The biggest value for the percentage of structural dependencies that are also co-changes is 76.5\% obtained in the case when no filterings are done.

From Tables \ref{tab:percLD:comm}, \ref{tab:percLD:nocomm}, and \ref{tab:percLDtrength} we notice that the percentage of co-changes which are also structural is always low to very low. This means that most co-changes are recorded between classes that have no structural dependencies to each other \cite{enase19}.   
 



\chapter{Conclusions}
\label{conclusions}

In order to obtain logical dependencies, we have to filter the co-changes extracted from the versioning system.
The filtering will increase the confidence that the remaining co-changing pairs are indeed logically coupled.
And also, will help to reduce the size of the co-changes extracted. This is an important aspect since surveys show that architectural reconstruction is mainly made based on structural dependencies \cite{Shtern:2012:CMS:2332427.2332428}, \cite{sar}, the main reason why historical information is not used in architectural reconstruction is the size of the extracted information.

Different applications based on dependency analysis could be improved if, beyond structural dependencies, they also take into account the hidden non-structural dependencies. For example, works  which investigate different methods for architectural reconstruction \cite{SoraConti}, \cite{SoraSem13}, \cite{PagerankENASE}, all of them based on the information provided by structural dependencies, could enrich their dependency models by taking into account also logical dependencies.  

The software architecture is important in order to understand and maintain a system. Often code updates are made without checking or updating the architecture. This kind of updates cause the architecture to drift from the reality of the code over time \cite{sar}.
So reconstructing the architecture and verifying if still matches the reality is important \cite{Kalliamvakou2016}. An increased knowledge of the software system help activities such as bug correction, enhancement, reuse and documentation \cite{Comprehension}, \cite{1199197}, \cite{2003:XLC:851042.857028}.

The usage of logical dependencies could have an impact also in fields like software comprehension \cite{article-cognitive-processes}, fault location \cite{fault-localization}, \cite{program-failures} and error proneness \cite{67595}, \cite{1702015}.



 %%%%%%%%%%%%%%%%%%%%%%%%%%%%%%%%%%%%%%%%%%%%%%%%%%%%%%%%%%%%%%%%%%%%%%%%%%%%%%%%


\bibliographystyle{plain}
\bibliography{logicaldepd}
\end{document}

